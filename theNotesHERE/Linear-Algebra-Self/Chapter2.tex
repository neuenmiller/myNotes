\chapter{Solving Linear Equation \(Ax = b\)}
\section{Elimination and Back Substitution}
I fucked up

\section{Elimination Matrices and Inverse Matrices}
This too.

\section{Matrix Computation and \(A = LU\)}
This one too.

\section{Permutation and Tranposes}
Also this shit.

\section{Derivatives and Finite Difference Matrices}
Second difference matrices includes \(K, T, B\).
They all have the \(-1, 2, -1\) pattern.
Now we can approximate \(-\frac{d^2 u}{dx^2} = f=(x)\) 

\[
K_4 =
\frac{1}{h^2}
    \begin{bmatrix}
        2 & -1 & 0 & 0  \\
         0 & -1 & 2 & -1  \\
         0 &  0 & -1  & 2  \\
        0 & 0 & -1 & 2  \\
    \end{bmatrix}
    \begin{bmatrix}
        u_1  \\
        u_2  \\
        u_3  \\
        u_4  \\
    \end{bmatrix}
    =
    \begin{bmatrix}
        f(h)  \\
        f(2h)  \\
        f(3h)  \\
        f(4h)  \\
    \end{bmatrix}
\]

So, we want to compute \(-\frac{d^2 u}{dx^2}\) with a computer, but the computer can't understand derivative. 
So what we do is we turn \(\frac{d^2 u}{dx^2}\)  into the matrix \(\frac{K^2}{h}\),
function \(u(x)\)  into vector \(u\), and function \(f(x)\) into \(F\).
We also need the boundary conditions, which are given where \(u(0) = 0 \text{ and } u(1)=0\). 
We can't pick out the infinite space between 0 and 1, so we pick N equally spaced points at a regular interval. 
The space between each points (and the first and the last point) becomes meshwidth (h). 
If we have N interal points \(u_0, u_1, u_2, \ldots\) plus two boundary points \(u_0 \text{ and } u_{N+1}\), we divide the total length into N+1 segments.
Therefore the spacing is \(h = \frac{1}{N+1}\).
If we have 4 N, then the spacing is h = \(\frac{1}{5}\).
So instead of finding the continuous function \(u(x)\), we will find the value at each internal points, and they becomes the unknown vector \(U = [u_1, u_2, u_3, u_4]^T\).
\medbreak
\(-\frac{d^2 u}{dx^2} \approx  \frac{-u(x+h) + 2u(x) - u(x-h)}{h^2}\) which means that it will
