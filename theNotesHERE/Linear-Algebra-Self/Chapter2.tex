\chapter{Solving Linear Equation \(Ax = b\)}
\section{Elimination and Back Substitution}
I fucked up

\section{Elimination Matrices and Inverse Matrices}
This too.

\section{Matrix Computation and \(A = LU\)}
This one too.

\section{Permutation and Tranposes}
Also this shit.

\section{Derivatives and Finite Difference Matrices}
Second difference matrices includes \(K, T, B\).
They all have the \(-1, 2, -1\) pattern.
\[
    K_4 =
     \begin{bmatrix}
        2 & -1 & 0 & 0  \\
        -1 & 2 & -1 & 0  \\
        0 &  -1 & 2 & -1  \\
        0 & 0 & -1 & 2  \\
    \end{bmatrix}
\]
Now we can approximate \(-\frac{d^2 u}{dx^2} = f=(x)\) 
So, we want to compute \(-\frac{d^2 u}{dx^2}\) with a computer, but the computer can't understand derivative. 
So what we do is we turn \(\frac{d^2 u}{dx^2}\)  into the matrix \(\frac{K^2}{h}\),
function \(u(x)\)  into vector \(u\), and function \(f(x)\) into \(F\).
We also need the boundary conditions, which are given where \(u(0) = 0 \text{ and } u(1)=0\). 
We can't pick out the infinite space between 0 and 1, so we pick N equally spaced points at a regular interval. 
The space between each points (and the first and the last point) becomes meshwidth (h). 
If we have N interal points \(u_0, u_1, u_2, \ldots\) plus two boundary points \(u_0 \text{ and } u_{N+1}\), we divide the total length into N+1 segments.
Therefore the spacing is \(h = \frac{1}{N+1}\).
If we have 4 N, then the spacing is h = \(\frac{1}{5}\).
So instead of finding the continuous function \(u(x)\), we will find the value at each internal points, and they becomes the unknown vector \(U = [u_1, u_2, u_3, u_4]^T\).
\[
-\frac{d^2 u}{dx^2} = f(x)
\text{ becomes }
\frac{KU}{h^2} = F,
\frac{1}{h^2}
    \begin{bmatrix}
        2 & -1 & 0 & 0  \\
        -1 & 2 & -1 & 0  \\
        0 &  -1 & 2 & -1  \\
        0 & 0 & -1 & 2  \\
    \end{bmatrix}
    \begin{bmatrix}
        u_1  \\
        u_2  \\
        u_3  \\
        u_4  \\
    \end{bmatrix}
    =
    \begin{bmatrix}
        f(h)  \\
        f(2h)  \\
        f(3h)  \\
        f(4h)  \\
    \end{bmatrix}
\]
The key point is that when divide the meshpoint into 4, therefore N = 4.
Row 1 times \(U\) is \(2u_1 - u_2\), and we already got the boundary where \(u_0 = 0\text{ and } u_5 = 0\), making a typical row \(\frac{(-u_1 + 2u_2 - u_3)}{h^2} = f(h)\). The division by \(h^2\) makes \(\frac{K}{h^2}\) a second difference matrix, replacing \(-\frac{d^2 u}{dx^2}\).   

\subsection{Properties of K}
K has 4 properties. For the sake of example, we will use \(K \text{ for } N = 4\). 
\[
    \begin{bmatrix}
        2 & -1 & 0 & 0  \\
        -1 & 2 & -1 & 0  \\
        0 &  -1 & 2 & -1  \\
        0 & 0 & -1 & 2  \\
    \end{bmatrix}
\]
\begin{enumerate}
    \item K is symmetrical, as in \(K_{ij} = K_{ji}\)
    \item K is banded. All the non-zeros (-1, 2, -1) lie in a band around the main diagonal. The band has three diagonals, so K is a tridiagonal matrix. 
    \item K has a constant diagonals. A diagonal of -1, then 2, then -1 again. The matrix is called "shift-invariant", because the differential equation always have a constant coefficient of -1. The approximation to \(-\frac{d^2 u}{dx^2}\)  is always \(-1, 2, -1 \text{ at every} X\).
    \item X is invertible. It has an inverse matrix \(K^{-1} \text{ then } K^{-1}K = I \text{ and } KK^{-1} = I\). 
    \[
        K_4^{-1} = \bm{\frac{1}{5}} \begin{bmatrix}
            4 & 3 & 2 & 1  \\
            3 & 6 & 4 & 2  \\
            2 & 4 & 6 & 3 \\
            1 & 2 & 3 & 4  \\
        \end{bmatrix}  
    \]  
    \(K^{-1} \) is aso symmetric but it is no diagonal. It is dense matrix, meaning no zeros. 
    \item Symmetric \(K_n\) matrices are positive definite.  
\end{enumerate}
Invertible, positive definite symmetric matrix, and semidefinite matrix are defined by their pivots.
\begin{enumerate}
    \item Invertible matrices has nonzero pivots.
    \item Positive definite symmetric matrices has positive nonzero pivots.
    \item Positive semidefinite symmetric matrices has nonnegative pivots.
\end{enumerate}

\subsection{Free-fixed Matrice \(T_n\)}

\(T_n \text{ and } B_n\) are variations on \(K_n\), where the variation comes from changing the boundary conditions.
Think of it as an elastic band that are fixed at both ends, one end, or totally free at both ends.  
For example, \(T_n\) is very similar to \(K_n\) except that input \((1, 1)\) is switched from 2 to 1., representing a free boundary condition where \(\frac{du}{dx} = 0\)
\[
    \text{Free-fixed boundary conditions, still positive definite }
    T_4 =
    \begin{bmatrix}
        1 & -1 & 0 & 0  \\
        -1 & 2 & -1 & 0  \\
        0 &  -1 & 2 & -1  \\
        0 & 0 & -1 & 2  \\
    \end{bmatrix}
\]    
T is no longer Toeplitz because its main constant, though it does have a simpler factorization than K; every pivot of T equals 1.
\[
    T =
    \begin{bmatrix}
        1 &  &  &   \\
        -1 & 1 &  &   \\
        0 & -1 & 1 &   \\
        0 & 0 & -1 & 1  \\
    \end{bmatrix}
    \begin{bmatrix}
        1 & -1 & 0 & 0  \\
         & 1 & -1 & 0 \\
         &  & 1 & -1  \\
         &  &  & 1  \\
    \end{bmatrix}
    = LU
\]

Note that \(U = L^T\).
Notice that U is a forward difference while L is a backward difference.
Together, they add up to be a second difference, meaning that \(x_{i+1} - 2x_i + x_{i-1}  \) correspond to [-1, 2, -1], meaning that T is a second difference. 

\subsection{The Free-Free Matrices \(B_n\) are Singular}

In this context, "singular" means "not invertible".
One test is simply seeing if determinant equals zero or not.

\begin{theorem}
    If B multiplies a nonzero vector \(x\) to produce \(Bx = 0\), then \(B\) can't be invertible.   
\end{theorem}

For example, free-free matrix has 1 (and not 2) in its (1, 1) and (3, 3) input.

\[
    B_3 =
    \begin{bmatrix}
        1 & -1 & 0  \\
        -1 & 2 & -1  \\
        0 & -1 & 1  \\
    \end{bmatrix}
    \text{ has }
    B_{3}x =
    \begin{bmatrix}
        1 & -1 & 0  \\
        -1 & 2 & -1  \\
        0 & -1 & 1  \\
    \end{bmatrix} 
    \begin{bmatrix}
        1 \\
        1 \\
        1 \\
    \end{bmatrix}
    =
    \begin{bmatrix}
        0  \\
        0  \\
        0  \\
    \end{bmatrix}
\]

\(B_n \text{ is singular while } T_n \text{ and } K_n\) are invertible because \(T_n \text{ and } K_n\) have a fixed end, which allows them to adjust, whereas \(B_n\) is free on both end so it can't adjust. 

