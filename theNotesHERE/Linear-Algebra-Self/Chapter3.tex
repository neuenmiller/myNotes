\chapter{The Four Fundamental Subspaces}

How can we define "vector space"? Well, if we are talking about \(R^3\), the key operation are \(v + w \text{ and } cv\).
Notice that v and w could be matrices, so we could have matrix spaces and function spaces. Then inside \(R^n\) we could only allow \(x\) that satisfies \(Ax = 0\), which will produce "nullspace of A".
All combination of solution to \(Ax = 0\) are also solutions, meaning that the nullspace is a subspace. To point it simply, nullspace is just \(x \text{ in } Ax \text{ that crushes it down to } 0\). Why is it called a space? Because it has structures and rules. 
\begin{enumerate}
    \item It must contain zero vector, \(A \times 0 = 0\) always.
    \item It must be closed under addition, meaning that if you take two vectors from the nullspace then add them together, they must still be in the nullspace.
    \item It must be closed under scalar addition, meaning that if you take any vector from the nullspace then multiplies them by a constant, the result must stil be in the nullspace. 
\end{enumerate}
Then, lastly, there are basis. A set of vectors that perfectly describes the space. Very important, some considers it the fundamental theorem. So, what it means is that basis are just sets of movement vectors for each dimension.
\begin{enumerate}
    \item For a 3D space, you will need forward, right, and up.
    \item For a 2D space, you will need right and up.
    \item For a 1D space (line of sight, nullspace), you only need the direction of the line.
\end{enumerate}

And so, \(n - r\) special solutions to \(Ax = 0\) are a basis for \(N(A)\). It's called rank-nullity theorem.
\[
    \text{Dimension of Input Space = Dimension of Column Space + Dimension of Nullspace, }
    n = r + (n - r)
\] 
Note that \(n - r\) is the dimension of the nullspace. 
One last note: THIS CHAPTER IS VERY IMPORTANT.
\section{Vector Spaces and Subspaces}

Here are a few fundamental points:
\begin{enumerate}
    \item All linear combination of \(cv + dw\) must stay in the vector space (What is a vector space? Very simply, all the possible spaces tha can be achieved by your given vector with vector addition or scalar multiplication), where \(c\) and \(d\) are scalar, and \(v \text{ and } w\) are vectors.
    \item The row space (meaning all possible linear combination of the row vectors) is "spanned" (made up of) rows of \(A\). While the column of \(A\) spans \(C(A)\).
    \item Matrices can be filled by more than just numbers. As long as it obeys the rules of a vector space, it can be treated as a vector. 
    For example, we have two equations, \(f(x) = x^2\) and \(g(x) = 2x\). Can they be added together? 
    Yes! \(h(x) = x^2 + 2\). Now if we fill it with the likes of \(\sin, \cos, x, x^2\), we can "span" and build a lot of other functions. 
    For example, all quadratic polynomials are "spanned" by the functions \(f_1(x) = 1\), \(f_2(x) = x\), and \(f_3(x) = x^2\)       
\end{enumerate}

\(R^n\) contains all column vector \(v\) to the length of \(n\). For this case, the components from \(v_1 \text{ to } v_n\) are all real numbers. However, if they allow for complex numbers \((i)\), the \(R^n \text{ becomes } C^n\). 
To reiterate, all linear combination of \(cv + dw\) must be in the vector space \(R^n\).
For example, all positive the set with all positive (meaning no vector consist of ANY nonpositive numbers) vectors \((v_1, \ldots, v_n)\) areNOT a vector space. 
Why? Simply take one simple vector, say \((1, 2)\) 
then multiply it by scalar of, say, c = -1. 
\((-1, -2)\) is NOT in our set, therefore it is not a vector space.
Or for another example, a set of solution for \(Ax = (1, \ldots, 1)\) is not a vector space because a line in \(R^n\) is not a vector space unless it goes the central point \((0, \ldots, 0)\).

\subsection{Examples of Vector Spaces}

Here are some examples of a neat vector space, the \(Z\) (zero vector) where \(0 = (0, 0, \ldots, 0)\). Combinations of \(c0 + d0\) are all still 0, so  still in the subspace.
How about vector space of matrices? We can do that. \(R^{3 \times 3} \) is a space that contains all \(3 \times 3\) matrices. It does satisfy all eight rules, so why not? It's also a vector space. 
How about a vector space of functions? Sure can. The line of functions \(y = ce^x\) (any c) is a line in a function space. 
This line contains all solutions to the differential equations of \(\frac{dy}{dx} = y\). Yet another function space contains all quadratics \(y = a + bx + cx^2\), where they are the solutions to \(\frac{d^3 y}{dx^3}\)
And to reiterate, space in this context means all possible linear combination of the vectors or matrices or functions, and they all stay inside it.

\subsection{Subspaces of Vector Spaces}

What ar subspaces? To put it simply, they are a flat plane inside the dimensional space, however, they are still the same dimension. Let's say, we got a \(R^3\) space. We can make a plane any way we want as long as it passes \((0, 0, 0)\), what we get may look like a 2D plane, but it's still 3D. Therefore, the plane is a subspace of  the full vector space \(R^3\).
\newline
Here is a list of possible subspaces of \(R^3\):
\begin{enumerate}
    \item Any line through \((0, 0, 0)\)
    \item Any plane through \((0, 0, 0)\)
    \item The whole space \(R^3\) 
    \item The zero vector \((0, 0, 0)\) 
\end{enumerate}

\subsection{The Column Space of A}

What we are trying to solve here is \(Ax = b\). We want to know b, right? Well, \(b\)  are a column space of \(A\). \(Ax\) is just a combination of A, and to get every possible \(b\), we need all possible \(x\), which is just all linear combination of \(A\), which is the column space of \(A\), as written earlier. To build on that, vector space is made up of column vectors.       

A crucial point to understand is that, \textbf{to solve \(Ax = b\) is just to express \(b\) as a combination of the columns}. \(b\) got to be in the column space of A, otherwise, it doesn't exist!

Caution: columns of A do not form a subspace. Neither do invertible matrices, or singular matrices. Only all linear combinations.

\subsection{The Row Space of A}

\textbf{The rows of A are the column of \(A^T\)}, why do we do this?
Because we like working with columns, so we use the column of \(A^T\)

\[
    \text{\textbf{The row space of A is just the column space of \(A^T\) }}
\]  



\section{Computing the Nullspace by Elimination: \(A = CR\) }

\begin{enumerate}
    \item The \textbf{nullspace N(A)} in \(\mathbf{R^n}\) contains all solutions x to \(Ax = 0\), including \(x = 0\).
    \item CONTINUE THIS LATER      
\end{enumerate}

The goal of this section is to find all solutions for \(Ax = 0\). 
If \(A\) is an invertible matrix, then the only solution is \(x = 0\). In general, A has \(r\)  independent columns, the other \(n - r\) are a linear combination. 
\newline
Here is a matrix \(R\) with rank \(r = 2\), with \(n = 4\) columns. This means we have \(n - r = 4 - 2 = 2\) independent solutions to \(Rx = 0\). So the nullspace \(N(R)\) will have 2 dimensions.
\[
    \text{\textbf{Example 1: }}
    R = [I F] P = \begin{bmatrix}
        1 & 0 & 3 & 5  \\
        0 & 1 & 4 & 6  \\
    \end{bmatrix}
\]

Which means \(Rx = 0\) is \(x_1 + 3x_3 + 5x_4 = 0\) and \(x_2 + 4x_3 + 6x_4 = 0\).
We find the special solutions by just letting \(x_3 \text{ and } x_4\) equal 1 and 0 or 0 and 1.
Set \(x_3 = 1, x_4 = 0\) the equations will give \(x_1 = -3, x_2 = -4\)    
Set \(x_3 = 0, x_4 = 1\) the equations will give \(x_1 = -5, x_2 = -6\)
This gives us two special solutions: \(s_1 = (-3, -4, 1, 0) \text{ and } s_2 = (-5, -6, 0, 1)\). They are both in the nullspace of \(R\), as we can also see, \(cs_1 + ds_2\) is still in the the nullspace. And so, \(s_1 \text{ and } s_2\) are the basis of nullspace.

\[
    \text{\textbf{Example 2: }} 
    R_0 =
    \begin{bmatrix}
        1 & 7 & 0 & 8  \\
        0 & 0 & 1 & 9  \\
        0 & 0 & 0 & 0  \\
    \end{bmatrix}
\]

Which means \(x_1 + 7x_2 + 0x_3 + 8x_4 = 0\), \(x_3 + 9x_4 = 0\), and \(0 = 0\) (wow on the last one). The matrix identity is inside column 1 and 3, and row 3 is all zero, which makes it a reduced row echelon form, even elimination can't make it simplier. We still have free variables for the special solution, namely, \(x_2 \text{ and } x_4\).   
Set \(x_2 = 1, x_4 = 0\), the equations give \(x_1 = -7, x_3 = 0\).
Set \(x_2 = 0, x_4 = 1\), the equations give \(x_1 = -8, x_3 = -9\).
The special solutions are noe \(s_1 = (-7, 1, 0, 0)\) and \(s_2 = (-8 ,0 , -9, 1)\).

First, we start with any \(m \text{ by } n\) matrix A, then apply elimination. That changes A into its reduced row echelon form, \(R_0 = \text{rref} (A)\). Removing all zero rows of \(R_0 \text{ leaves } R\).
\[
    r, m, n = 2, 2, 4 \text{ Simplest Case } R = [I F]
    \text{ as in}
    \begin{bmatrix}
        1 & 0 & 3 & 5  \\
        0 & 1 & 4 & 6  \\
    \end{bmatrix}
\]    

\[
    r, m, n = 2, 3, 4 \text{ General Case } 
    R_0 =
    \begin{bmatrix}
        I & F  \\
        0 & 0  \\
    \end{bmatrix}
    P
    \text{ as in }
    \begin{bmatrix}
        1 & 7 & 0 & 8  \\
        0 & 0 & 1 & 9  \\
        0 & 0 & 0 & 0  \\
    \end{bmatrix}
\] 
Hold up, what is rref, I, F here?
For something to be rref, you need to satisfy 4 conditions.
\begin{enumerate}
    \item The zero row is at the bottom
    \item The first non-zero in row 1 is 1, the first non-zero entry in row 2 is 1. 
    \item Each pivot is to the right of the pivots in the rows above it.
    \item Every pivot is only non-zero number in its entire column.
\end{enumerate}

\(I\) in this context is still identity, but only if you take the pivot columns and align them chronologically as they were.
\(F\) is free matrix, the other non-pivot columns, still in the same chronologically order.

In this case \((R_0)\),
\[
    \text{I is }
    \begin{bmatrix}
        1 & 0  \\
        0 & 1  \\
        0 & 0  \\
    \end{bmatrix}
    , 
    \text{ F is }
    \begin{bmatrix}
        7 & 8  \\
        0 & 9  \\
        0 & 0  \\
    \end{bmatrix}
\]

\subsection{Elimination from \(A\)  to rref \((A)\): Reduced Row Echelon Form}

Refresher, how does elimination works?
\begin{enumerate}
    \item Subtruct a multiple of one row from another row
    \item Multiple a row by nonzero number
    \item Exchange any rows
\end{enumerate}
For demonstration,
\[
    A =
    \begin{bmatrix}
        1 & 2 & 11 & 17  \\
        3 & 7 & 37 & 57  \\
    \end{bmatrix}
    \text{ then }
    \begin{bmatrix}
        1 & 2 & 11 & 17  \\
        0 & 1 & 4 & 6  \\
    \end{bmatrix}
    \text{ then }
    \begin{bmatrix}
        1 & 0 & 3 & 5  \\
        0 & 1 & 4 & 6  \\
    \end{bmatrix}
\]
So, what did elimination actually do? It inverted the leading 2 by 2 matrix, which we will call W.
\[
    W =
    \begin{bmatrix}
        1 & 2  \\
        3 & 7  \\
    \end{bmatrix}
    \text{ into }
    \begin{bmatrix}
        1 & 0  \\
        0 & 1  \\
    \end{bmatrix}
\]
We multiplied \(W^{-1}A = W^{-1}[ W H ] \text{ to produce } R =  [I W^{-1} H ] = [ I F ]  \). We always knew that free columns \((H)\) is some combination of independent columns \((W)\), but we now know that \(H = W F\).
\[
    H =
    \begin{bmatrix}
        11 & 17 \\
        37 & 57  \\
    \end{bmatrix}
    = W F =
    \begin{bmatrix}
        1 & 2  \\
        3 & 7  \\
    \end{bmatrix}
    \times
    \begin{bmatrix}
        3 & 5  \\
        4 & 6  \\
    \end{bmatrix}
\]
\textbf{However you compute R from A, you will always get the same R. R is completely controlled by A.}

For \textbf{example 2}, let us rref another \(A\). 
\[
    A = 
    \begin{bmatrix}
        1 & 7 & 3 & 35  \\
        2 & 14 & 6 & 70  \\
        2 & 14 & 9 & 97  \\
    \end{bmatrix}
    \rightarrow
    \begin{bmatrix}
        1 & 7 & 3 & 35  \\
        0 & 0 & 0 & 0  \\
        0 & 0 & 3 & 27  \\
    \end{bmatrix}
    \rightarrow
    \begin{bmatrix}
        1 & 7 & 0 & 8  \\
        0 & 0 & 0 & 0  \\
        0 & 0 & 3 & 27  \\
    \end{bmatrix}
    \rightarrow
    \begin{bmatrix}
        1 & 7 & 0 & 8  \\
        0 & 0 & 1 & 9  \\
        0 & 0 & 0 & 0  \\
    \end{bmatrix}
    = R_0
\] 

\subsection{Elimination Column by Column: The Steps from A to \(R_0\) }

We will now reduce what we learn to an easily applied algorithm. 
The big question is \textbf{does this new column \(k + 1\) join with \(I_k \text{ or } F_k\)?}

\textbf{If \(l\) is all zero }, the new column is \textbf{dependent} on the first \(k\) columns. Then \(\mathbf{u} \) joins with \(F_k\) to become \(F_{k+1} \). 

\textbf{If \(l\) is not all zero }, then it is independent of the first k columns. Use the \textbf{largest} number, preferably, in \(l\) as the pivot. \textbf{Important} thing to remember here is that the column are talking about means the columns \textbf{UNDERNEATH} all pivots, not \textbf{ALL} columns. Then do elimination. Whatever is left becomes part of \(\mathbf{I} \) as \(\mathbf{I_{k+1} }  \). 
 
From example 2, we can see that the combination of independent and dependent comes out to 
\[
    C \times F = 
    \begin{bmatrix}
        1 & 3  \\
        2 & 6  \\
        2 & 9  \\
    \end{bmatrix}
    \begin{bmatrix}
        7 & 8  \\
        0 & 9  \\
    \end{bmatrix}
    = 
    \begin{bmatrix}
        7 & 35  \\
        14 & 70  \\
        14 & 97  \\
    \end{bmatrix}
    =
    \text{depend columns of 2 and 4 of }A
\]

Right back to where we came from, showing that \(C\) are almost like the ingredients and \(F\) are almost like the method. 

\subsection{The Matrix Factorization \(A = CR\) and the Nullspace}

In chapters prior, we know that \(A = CR\) but we have no systemic way to find them, now we do. We apply elimination to reduce \(A \text{ to } R_0\). Then \(I\) in \(R_0\) locates the matrix C of independent columns in \(A\). Removing zero row in \(R_0\) produces \(R\) for \(A = CR\)

We have two special solution \(s_1\) and \(s_2\) for every column of \(F\) in \(R\).

\[
    Rs_1 = 0,
    \begin{bmatrix}
        1 & 7 & 0 & 8  \\
        0 & 0 & 1 & 9  \\
    \end{bmatrix}
    \begin{bmatrix}
        -7 \\
        1  \\
        0  \\
        0  \\
    \end{bmatrix}
    = 
    \begin{bmatrix}
        0 \\
        0 \\
    \end{bmatrix}
\]
\[
    Rs_1 = 0,
    \begin{bmatrix}
        1 & 7 & 0 & 8  \\
        0 & 0 & 1 & 9  \\
    \end{bmatrix}
    \begin{bmatrix}
        -8 \\
        0  \\
        -9  \\
        0  \\
    \end{bmatrix}
    = 
    \begin{bmatrix}
        0 \\
        0 \\
    \end{bmatrix}
\]
\(s_1\) and \(s_2\) are the easiest to see using the matrices \(-F\) and \(I\) and \(P^T\). 

\[
    \text{As a reminder, the two special solutions to }
    [I F] P x = 0
    \text{ are the columns of }
    P^T 
    \begin{bmatrix}
         -F \\
          I \\
    \end{bmatrix}
\]
It more correct than the other option because \(P P^T\) is the identity matrix of permutation matrix P:
\[
    \mathbf{Rx = 0, }
    [I F] P 
    \times
    P^T 
    \begin{bmatrix}
         -F \\
         I \\
    \end{bmatrix}
    \text{ reduces to }
    [I F]
    \begin{bmatrix}
         -F \\
          I \\
    \end{bmatrix}
    = 
    [0]
\] 

\textbf{Review} Say, the \(m\) by \(n\) matrix \(A\) has rank \(r\). We can find \(n - r\) special solution to \(Ax = 0\)  by computing the rref \(R_0\) of \(A\). Remove the \(m - r\) zero rows of \(R_0\) to produce \(R = [I F] P\) and \(A = CR\). Then the special solutions to \(Ax = 0\) are the \(n - r\) columns of \(P^T [-F, I]^T\)  

\textbf{Example 3: Elimination of \(A\) gives \(R_0\) and \(R\). \(R\) reveals the nullspace of \(A\) } 
\[
    A =
    \begin{bmatrix}
        1 & 2 & 1  \\
        2 & 4 & 5  \\
        3 & 6 & 9  \\
    \end{bmatrix}
    \rightarrow
    \begin{bmatrix}
        1 & 2 & 1  \\
        0 & 0 & 3  \\
        0 & 0 & 6  \\
    \end{bmatrix}
    \rightarrow
    \begin{bmatrix}
        1 & 2 & 0  \\
        0 & 0 & 1  \\
        0 & 0 & 0  \\
    \end{bmatrix}
    = R_0
    \text{ with rank 2}
\]
\[
    R = 
    \begin{bmatrix}
        1 & 2 & 0  \\
        0 & 0 & 1  \\
    \end{bmatrix}
\]
The independent columns are 1 and 3. 

To solve \(Ax = 0\) and \(Rx = 0\), set \(x_2 = 1\), which will get you \(x_1 = -2, x_3 = 0\). Leave us special solution:
\[
    \mathbf{s = (-2, 1, 0)}  
\]  
All solutions \(x = (-2c, c, 0)\). And here it is, \(A = CR\)
\[
     A =
    \begin{bmatrix}
        1 & 2 & 1  \\
        2 & 4 & 5  \\
        3 & 6 & 9  \\
    \end{bmatrix}
    = CR =
    \begin{bmatrix}
        1 & 1  \\
        2 & 5  \\
        3 & 9  \\
    \end{bmatrix}
    \begin{bmatrix}
        1 & 2 & 0  \\
        0 & 0 & 1  \\
    \end{bmatrix}
    = 
    \text{columns basis in \(C\) } \times \text{row basis in \(R\).}
\]  

For a lot matrices, the only solution to \(Ax = 0\) is \(x = 0\). Simply, all columns of \(A\) are independent. The nullspace \(N(A)\) contains only the zero vector, no special solution. This case zero nullspace is \textbf{important} because it means that all columns of \(A\) is independent. But this can't happen if \(n > m\) (column > row) because you can have \(n\) independent column in \(R^m\).   

\textbf{Important} Say \(A\) has more columns than rows \((n > m)\), there will be at least one free variable. Meaning that \(Ax = 0\) has at least one non-zero solution. Or to put it more specifically, there must be more than \(n - m\) free columns. \(Ax = 0\) must have nonzero solutions in \(N(A)\). 

\medbreak

\textbf{Example 4: Find the nullspace of A, B, M and the two special solutions to \(Mx = 0\) }
\[
    A = 
    \begin{bmatrix}
        1 & 2  \\
        3 & 8  \\
    \end{bmatrix}
    ,  
    B =
    \begin{bmatrix}
         A \\
         2A \\
    \end{bmatrix}
    = 
    \begin{bmatrix}
        1 & 2  \\
        3 & 8  \\
        2 & 4  \\
        6 & 16  \\
    \end{bmatrix}
    ,
    M = 
    \begin{bmatrix}
        A & 2A  \\
    \end{bmatrix}
    = 
    \begin{bmatrix}
        1 & 2 & 2 & 4  \\
        3 & 8 & 6 & 16  \\
    \end{bmatrix}
\]
\textbf{Solution} The equation \(Ax = 0\) has only the zero solution \(x = 0\). The nullspace is only \(Z\). 
\[
    Ax = 
    \begin{bmatrix}
        1 & 2  \\
        3 & 8  \\
    \end{bmatrix}
    \rightarrow
    \begin{bmatrix}
        1 & 2  \\
        0 & 2  \\
    \end{bmatrix}
    \rightarrow
    \begin{bmatrix}
        1 & 0  \\
        0 & 1  \\
    \end{bmatrix}
    = R = I
\]   

No free variables, meaning A is invertible; therefore, no special solution. 

The M matrix is different. It has extra columns instead of rows. That means that, with 4 columns and 2 rows, there will be 2 free columns leftover. 
\[
    M = 
    \begin{bmatrix}
        1 & 0 & 2 & 0  \\
        3 & 8 & 6 & 16  \\
    \end{bmatrix}, 
    R = 
    \begin{bmatrix}
        1 & 0 & 2 & 0  \\
        0 & 1 & 0 & 2  \\
    \end{bmatrix}
    = 
    \begin{bmatrix}
        I & F  \\
    \end{bmatrix}
\]
Again, to get special solutions out of it, we will let \(x_3 = 1, x_4 = 0\) and \(x_3 = 0, x_4 = 1\). What we will get is the special solution for the nullspace of \(M\). 
\[
    Mx = 0
    R = 
    \begin{bmatrix}
        1 & 0 & 2 & 0  \\
        0 & 1 & 0 & 2  \\
    \end{bmatrix}
    s_1 = 
    \begin{bmatrix}
        -2 \\
        0 \\
        1\\
        0\\
    \end{bmatrix}
    \text{ and }
    s_2 = 
    \begin{bmatrix}
         0 \\
         -2 \\
         0 \\
         1 \\
    \end{bmatrix}
\]  

\subsection{Block Elimination in Three Steps: Final Thoughts}

We will conclude nicely with three steps to block elimination.

\textbf{Step 1} Exchange the columns and rows of \(P_C\) and \(P_R\) so that that \(r\) independent columns and rows come first in \(P_{R}AP_C \)
\[
    P_{R}AP_C = 
    \begin{bmatrix}
        W & H  \\
        J & K  \\
    \end{bmatrix}
    , C = 
    \begin{bmatrix}
         W \\
         J \\
    \end{bmatrix}
    \text{ and }
    , B = 
    \begin{bmatrix}
        W & H  \\
    \end{bmatrix}
\]    

\textbf{Step 2} Multiple the top rows by \(W^{-1}\) to produce \(W^{-1}B = [I, W^{-1}H] = [I, F]\).

\textbf{Step 3} Subtract \(J[I, W^{-1}H]\) from \([J, K]\) to produce \([0, 0]\). 

\textbf{The results of the steps should be an rref form of \(R_0\)}
\[
    P_{R}AP_{C} = 
    \begin{bmatrix}
        W & H  \\
        J & K  \\
    \end{bmatrix}
    \rightarrow
    \begin{bmatrix}
        I & W^{-1}H  \\
        J & K  \\
    \end{bmatrix}
    \rightarrow
    \begin{bmatrix}
        I & W^{-1}H   \\
        0 & 0  \\
    \end{bmatrix}
    = R_0
\] 

There are two things that need to be remembered. 
\begin{enumerate}
    \item \(W\)  is invertible
    \item The block satisfies \(JW^{-1}H = K\) 
\end{enumerate}

1. We must think back to \(A = CR\). We can see that \(B = WR\), and since \(B\) and \(R\) have the rank of \(r\) and \(W\) is also \(r \text{ by } r\), that means that \(W\) must have a rank of \(r\) and be invertible. 

2. We know that the first row \([I, W^{-1}H]\) is linearly independent. Since \(A\) has the rank \(r\), it means that the lower row \([J, K]\) must be a combination of the upper rows. This means for the combination to be valid, \(JI = J\) and \(JW^{-1}H = K\). 
\[
    \text{The conclusion is that }
    P_{R}AP_{C} = 
    \begin{bmatrix}
         W \\
         J \\
    \end{bmatrix}
    W^{-1}
    \begin{bmatrix}
        W & H  \\
    \end{bmatrix}
    = CW^{-1}B.
\] 

\section{The Complete Solution to \(Ax = b\)}

Our goal in this section will be about: 
\begin{enumerate}
    \item The complete solution to \(Ax = b\): \(x = x_p + x_n\), where p starts for any particular x and \(n\) nullspace. 
    \item Elimination from \(Ax = b\) to \(R_{0}x = d\): Solvable when zero rows of \(R_0\) have zero in \(d\).   
    \item When \(R_{0}x = d\) is solvable, one \(x_p\) has all free variable equal to zero. 
    \item A has full column rank \(r = n\) when its nullspace \(N(A)\) = zero vector: no free variables. 
    \item A has full row rank \(r = m\) when its column space \(C(A)\) is \(R^m\): \(Ax = b\) is always solvable.         
\end{enumerate}

The biggest thing that changed is that the \(b\) of \(Ax = b\) is now not zero. Therefore, the row operation will also act on the right side, the \(b\) side. \(Ax = b\) is reduced to a simpler \(R_{0}x = d\) with the same, if any, solutions. One way to organize that is by augmenting (adding another column to the right side of the matrix). We can augment \(A\) with the right side of \((b_1, b_2, b_3) = (1, 6, 7)\) to produce augmented matrix \([A b]\)         

\[
    \begin{bmatrix}
        1 & 3 & 0 & 2  \\
        0 & 0 & 1 & 4  \\
        1 & 3 & 1 & 6  \\
    \end{bmatrix}
    \begin{bmatrix}
         x_1 \\
        x_2  \\
        x_3  \\
        x_4  \\
    \end{bmatrix}
    = 
    \begin{bmatrix}
         1 \\
         6 \\
         7 \\
    \end{bmatrix}
    \text{ has the augmented matrix}
    \begin{bmatrix}
        1 & 3 & 0 & 2  & 1  \\
        0 & 0 & 1 & 4 & 6  \\
        1 & 3 & 1 & 6 & 7  \\
    \end{bmatrix}
    = [A b]
\]

Now if we turn it into its rref form 

\[
    \begin{bmatrix}
        1 & 3 & 0 & 2  \\
        0 & 0 & 1 & 4  \\
        0 & 0 & 0 & 0  \\
    \end{bmatrix}
    \begin{bmatrix}
         x_1 \\
        x_2  \\
        x_3  \\
        x_4  \\
    \end{bmatrix}
    = 
    \begin{bmatrix}
         1 \\
         6 \\
         0 \\
    \end{bmatrix}
    \text{ has the augmented matrix}
    \begin{bmatrix}
        1 & 3 & 0 & 2  & 1  \\
        0 & 0 & 1 & 4 & 6  \\
        0 & 0 & 0 & 0 & 0  \\
    \end{bmatrix}
    = [R_0 d]
\]
The last row is very important. Third equation became 0 = 0, each means it can be solved. In the original matrix, the first row plus the second row equals the third row. Meaning to solve \(Ax = b\) we need \(b_1 + b_2 = b_3\), which led to \(0 = 0\) in the third equation. 

\subsection{One Particular Solution \(Ax_p = b\) }

To get an easy \(x_p\), let the free variables be zeros: \(x_2 = x_4 = 0\), and the two nonzero equations be the two pivot variables \(x_1 = 1, x_3 = 6\). So our \(x_p\) is \(x_p = (1, 0, 6, 0)\). To put it simply, \textbf{free variables = zero, pivots = variable from d}. 

For a solution to exist, zero rows in \(R_0\) must also be zero in \(d\). 

\[
    R_{0}x_p = 
    \begin{bmatrix}
        1 & 3 & 0 & 2  \\
        0 & 0 & 1 & 4  \\
        0 & 0 & 0 & 0  \\
    \end{bmatrix} 
    \begin{bmatrix}
         1 \\
         0 \\
          6\\
          0\\
    \end{bmatrix}
    = 
    \begin{bmatrix}
         1 \\
          6\\
          0\\
    \end{bmatrix}
\]

Notice how the complete solution includes all \(x_n\):
\[
    x = x_p + x_n = 
    \begin{bmatrix}
         1 \\
         0 \\
         6 \\
         0 \\
    \end{bmatrix}
    + x_2
    \begin{bmatrix}
         -3 \\
         1 \\
         0 \\
         0 \\
    \end{bmatrix}
    +
    \begin{bmatrix}
        -2  \\
        0  \\
        -4  \\
        1  \\
    \end{bmatrix}
\] 

\textbf{Example 1}
Find the condition on \(b_1, b_2, b_3\) for \(Ax = b\) for 
\[
    A = 
    \begin{bmatrix}
        1 & 1  \\
        1 & 2  \\
        -2 & -3  \\
    \end{bmatrix}
    \text{ and }
    \begin{bmatrix}
         b_1 \\
         b_2 \\
         b_3 \\
    \end{bmatrix}
\]   
\textbf{Solution} Elimination using augmented matrix \([A, b]\)
\[
    \begin{bmatrix}
        1 & 1 & b_1  \\
        1 & 2 & b_2  \\
        -2 & -3 & b_3  \\
    \end{bmatrix}
    \rightarrow
    \begin{bmatrix}
        1 & 1 & b_1  \\
        0 & 1 & b_2 - b_1  \\
        0 & -1 & b_3 + 2b_1  \\
    \end{bmatrix}
    \rightarrow
    \begin{bmatrix}
        1 & 0 & 2b_{1} - b_2  \\
        0 & 1 & b_2 - b_1  \\
        0 & 0 & b_3 + b_1 + b_2  \\
    \end{bmatrix}
    = 
    [R_0 d ]
\] 

And since there is no special solution \((n - r = 2 - 2 = 0)\), the nullspace solution is \(x_n = 0\). So the complete solution is 
\[
    x = x_p + x_n = 
    \begin{bmatrix}
         2b_1 - b2 \\
         b_2 - b_1 \\
    \end{bmatrix}
    + 
    \begin{bmatrix}
         0 \\
         0 \\
    \end{bmatrix}
\]  

\textbf{Every matrix \(A\) with full column rank \((r = n) has these properties:\)  }
\begin{enumerate}
    \item All columns are independent, no free variables. 
    \item The nullspace \(N(A)\) include only the zero vector \(x = 0\)
    \item If \(Ax = b\) has a solution, then it has only one solution.    
\end{enumerate} 
With full column rank, \(Ax = b\) will have \textbf{one or no solution only.}

\subsection{Full Row Rank and the Complete Solution}

Another extreme case is the full row rank. Now \(Ax = b\) has one or infinitely many solutions. Full row rank requires that the matrix be a short and wide one (m > n, row q > col q), and every row is independent. 
For \textbf{example 2}, \(Ax = b\) has 3 n but only two m. 
\[
    \begin{bmatrix}
        1 & 1 & 1 & 3  \\
        1 & 2 & -1 & 4  \\
    \end{bmatrix}
    \text{ rank r = m = 2}
\]   

Imagine them as plane in \(xyz\) space. There are two planes, colliding into a line. The particular solution is a spot on that line, and the nullspace vector will move us along that line. \(x = x_p + \text{ all } x_n\) gives us the whole line solution. 

Fast forward a bit, getting ex. 2 into the \([A b]\) form gives us:
\[
    \begin{bmatrix}
        1 & 0 & 3 & 2  \\
        0 & 1 & -2 & 1  \\
    \end{bmatrix}
    = [R, d]
\] 
This particular solution \((2, 1, 0)\) has free variable \(x_3 = 0\). The special solution has \(s_3 = 1\), and the \(-x_1\) and \(-x_2\) comes from the free column of R. 

Check that \(x_p\) and \(s\) satisfies \(Ax_p = b\) and \(As = 0\)
\[
    2 + 1 = 3, 2 + 2 = 4
    -3 + 2 + 1 = 0, -3 + 4 - 1 = 0
\]    
Remove that the nullspace solution \(x_n\) is just any multiple of \(s\). 
\[
    \text{Computer solution: } 
    x = x_p + x_n = 
    \begin{bmatrix}
         2 \\
         1 \\
         0 \\
    \end{bmatrix}
    + x_3
    \begin{bmatrix}
         -3 \\
         2 \\
         1 \\
    \end{bmatrix}
\]  

Every matrix \(A\) with a full row rank \((r = m)\) has all these properties: 
\begin{enumerate}
    \item All rows have pivot and \(R_0\) has no zero rows: \(R_0 = R\)
    \item \(Ax = b\) has a solution for every right side \(b\).
    \item The column space of \(A\) is the whole space \(R^m\).
    \item If \(m < n\) , the equation has many solutions (called underdetermined in formal language).    
    \item The rows are linearly independent.  
\end{enumerate}  

There are \textbf{four} possibilities for linear equation depend on rank \(r\)
\begin{enumerate}
    \item \(r = m\) and \(r = n\) Square ad invertible, \(Ax = b\) has 1 solution. 
    \item \(r = m\) and \(r < n\) Short and wide, \(Ax = b\) has infinite solutions.    
    \item \(r < m\) and \(r = n\) Tall and thin, \(Ax = b\) has 0 or 1 solutions. 
    \item \(r < m\) and \(r < n\) Not full rank, \(Ax = b\) has 0 or infinite solutions.        
\end{enumerate} 

The reduced \(R_0\) will fall in the same category as matrix A. For \(R_{0}x = d\) and \(Ax = b\) to be solvable, \(d\) must end in \( m - r\) zeros. 

\textbf{Four types for \(R_0\) }
\[
    \begin{bmatrix}
         I \\
    \end{bmatrix}
    , r = m = n
\] 
\[
    \begin{bmatrix}
        I & F  \\
    \end{bmatrix}
    r = m < n 
\]
\[
    \begin{bmatrix}
         I \\
         0 \\
    \end{bmatrix}
    r = n < m
\]
\[
    \begin{bmatrix}
        I & F  \\
        0 & 0  \\
    \end{bmatrix}
    r < m, r , n
\]

Case 1 and 2 have full row rank \(r = m\). Case 1 and 3 have full column rank \(r = n\).  

\section{Independence, Basis, Dimension}

How big is the true size of a subspace? A matrix might be \(m\) by \(n\) but the column space is not necessarily \(n\). The column space \(C(A)\) is measured by independent columns. This will be clarified later. 

Our goal here is to understand a \textbf{basis: independent vectors that "spans" a space.} Every vector in the space is an unique combination of the basis vectors. 
Some vague explanation of the terms:
\begin{enumerate}
    \item Independent vectors, no extra vectors 
    \item Spanning a space, enough vectors to produce the rest. 
    \item Basis for a space, not too many and not too few 
    \item Dimension of a space, the number of vectors in every basis. 
\end{enumerate}


\subsection{Linear Independence}
Here's an odd definition for independence:

\[
    \text{The columns of \( A \) are linearly independent when the only solution of \( Ax = 0 \) is \( x = 0 \). No other combination of \( Ax \) gives the zero vector.}
\]

To illustrate why: it is impossible for three vectors that are not in a plane to be in a plane \textbf{unless} \( x \) is \( 0 \), i.e.,
\[
    0v_1 + 0v_2 + 0v_3 + \ldots
\]

Or to put it in other words:
\[
    \text{Linear independence only happens when } x_{1}v_1 + x_{2}v_2 + \ldots = 0 \text{ and all } x_i = 0.
\]

One point to drive is that vectors are either dependent or independent, no in between. For some examples:
 \begin{enumerate}
    \item \((1, 0)\) and \((1, 0.000001)\) are independent.
    \item \((1, 1)\) and \((-1, -1)\) are dependent.
    \item \((1, 1)\) and \((0, 0)\) are dependent because of the zero vector.
    \item In \(\mathbb{R}^2\): Any three vectors \((a, b), (c, d), (e, f)\) are dependent.
\end{enumerate}

\textbf{Example 1: } The columns of this A are dependent, \(Ax = 0\)  has a nonzero solution: 
\[
    Ax = 
    \begin{bmatrix}
        1 & 0 & 3  \\
        2 & 1 & 5  \\
        1 & 0 & 3  \\
    \end{bmatrix}
    \begin{bmatrix}
         -3 \\
         1 \\
         1 \\
    \end{bmatrix}
    \text{ is }
    -3 \begin{bmatrix}
         1 \\
         2 \\
         1 \\
    \end{bmatrix}
    +
    1 \begin{bmatrix}
         0 \\
         1 \\
         0 \\
    \end{bmatrix}
    +
    1 \begin{bmatrix}
         3 \\
         5 \\
         3 \\
    \end{bmatrix} 
    = 
    \begin{bmatrix}
         0 \\
         0 \\
         0 \\
    \end{bmatrix}
\]  

\textbf{Question}, how do we fix this? Just do elimination.
\[
    A = 
    \begin{bmatrix}
        1 & 0 & 3  \\
        2 & 1 & 5  \\
        1 & 0 & 3  \\
    \end{bmatrix}
    \rightarrow
    R_0 = 
    \begin{bmatrix}
        1 & 0 & -3  \\
        0 & 1 & -1  \\
        0 & 0 & 0  \\
    \end{bmatrix}
    . 
    F = 
    \begin{bmatrix}
         3 \\
         -1 \\
    \end{bmatrix}
    , x = 
    \begin{bmatrix}
         -3 \\
         1 \\
         1 \\
    \end{bmatrix}
\]

Remember that \textbf{columns of A are independent exactly when r = n}, there are \(n\) pivots and zero free variable. \(x = 0\) is the only nullspace. 

Also remember that any set of \(n\) vectors must be linearly dependent if \(n > m\). 

\subsection{Vectors that Span a Subspace}

To reiterate, column space consists of all combination of \(x_{1}v_1 + \ldots + x_{n}v_n\). The word "span" describes \(C(A)\). 
\[
    \text{The columns of a matrix span its column space. They might be dependent.}
\]  

\textbf{Example 2:} We will try to describe the column space and row space of A. 
\[
    m = 3, n = 2: 
    A = 
    \begin{bmatrix}
        1 & 4  \\
        2 & 7  \\
        3 & 5  \\
    \end{bmatrix}
    \text{ and }
    A^T = 
    \begin{bmatrix}
        1 & 2 & 3  \\
        4 & 7 & 5  \\
    \end{bmatrix}
\] 
We could say that the column space of A is a plane in \(R^3\) spanned by two columns of \(A\) The row space of A is spanned by three rows of A (which are columns in \(A^T\)) in \(R^2\). Oh, and the rows span \(R^n\) and columns span \(R^m\). Yes, they are swapped. 

\subsection{A Basis for a Vector Space}
To reiterate, the basis means just right. Two independent vectors can't span \(R^3\), four vectors can't be all independent even if they span \(R^3\). Three independent vectors for \(R^3\) is just right. 
\[
    \text{Basis vectors are independent and they span the space}
\]    

Note that there is \textbf{only one way to write \(v\) as a combinaion of the basis vector}

\textbf{Example 3:} The column of \(I = \begin{bmatrix}
    1 & 0  \\
    0 & 1  \\
\end{bmatrix}\) produce the standard basis for \(\mathbf{R^2}\). 
\[
    i = 
    \begin{bmatrix}
         1 \\
         0 \\
    \end{bmatrix}
    \text{ and }
    \begin{bmatrix}
         0 \\
         1 \\
    \end{bmatrix}
    \text{ are independent, and they span \(R^2\)}
\] 

\textbf{Example 4:} The column of every invertible \(n\) by \(n\) matrix give a basis for \(R^n\):
\[
    A =
    \begin{bmatrix}
        1 & 0 & 0  \\
        1 & 1 & 0  \\
        1 & 1 & 1  \\
    \end{bmatrix}
    , B = 
    \begin{bmatrix}
        1 & 0 & 1  \\
        1 & 1 & 2  \\
        1 & 1 & 2  \\
    \end{bmatrix}
\]    

We can see that A is \textbf{invertible}, \textbf{independent}, with a rank = 3. B is singular matrix, which also have dependent columns, and it is not full column rank (column space \(\neq  R^3\) ). The only solution to \(Ax = 0\) is \(x = A^{-1}0 = 0\). However, for \(Ax = b\) can always be solved by \(x = A^{-1}b\) Everything comes together for invertible matrices. 
 \textbf{The vector \(v_1, \ldots, v_n\) are a basis for \(R^n\) exactly when they are the column of an \(n\) by \(n\) invertible matrix. \(R^n\) also has infinitely many bases}
 Or to put it more compactly, \textbf{every set of independent vectors can be extended to a basis; the spanning set of vectors can be reduced to a basis.} 

 \textbf{Example 5:} The matrix is not invertible so its column are not basis for anything. 
 \[
    A = 
    \begin{bmatrix}
        2 & 4  \\
        3 & 6  \\
    \end{bmatrix}
    \rightarrow
    R_0 = 
    \begin{bmatrix}
        1 & 2  \\
        0 & 0  \\
    \end{bmatrix}
 \] 

 \textbf{Example 6:} Find the bases for the column and row spaces of this rank two matrix. 
 \[
    R_0 = 
    \begin{bmatrix}
        1 & 2 & 0 & 3  \\
        0 & 0 & 1 & 4  \\
        0 & 0 & 0 & 0  \\
    \end{bmatrix}
 \] 

Like us see, so for column space of \(R_0\), we got two pivots in a \(R^3\), where \(m = 3\); we got a subspace in \(R^3\)  And for row space, we got two non-zero rows and a rank of \(R^4\) where \(n = 4\). 

\textbf{Question!} Given five vectors in \(R^7\), how do we find their a basis that they span on? 
Two ways,
\begin{enumerate}
    \item Make them into a row and eliminate to find their non-zero. 
    \item Make them into columns, and eliminate to find pivots. The pivot columns are the basis. 
\end{enumerate}  

Another question, can another basis for the same vector space have more or less vectors? Answer, no! Number of vectors are tied to the space. 

\subsection{Dimension of a Vector Space}

This chapter is proof heavy, but the gist of it is simply that there can be many vector bases to choose from but no matter what, to the same vector space, there will always be the same amount of vectors. 

That is what dimension means, the number of basis vectors. 

\subsection{Bases for Matrix Spaces and Function Spaces}

We should also know that \textbf{independence}, \textbf{basis}, \textbf{dimension} are not just limited to column vectors. We can also ask whether specific matrices \(A_1, A_2, A_3\) are independent or not. Say, a \(3 \times 4\) matrix space, what dimension is that? 12 dimensions since you need twelve matrices in every basis. 

In differential equation, \(\frac{d^{2}y}{dx^2} = y\) has a space of solutions. One of the basis is \(y = e^x\) and \(y = e^{-x}\). Oh, and the basis function gives dimension = 2 for this solution space because the linear equation starts with the second derivative. 

\textbf{Matrix space} The vector space \(\mathbf{M} \) contains all 2 by 2 matrices, it's dimension is 4. 
\[
    A_1, A_2, A_3, A_4 = 
    \begin{bmatrix}
        1 & 0  \\
        0 & 0  \\
    \end{bmatrix}
    \begin{bmatrix}
        0 & 1  \\
        0 & 0  \\
    \end{bmatrix}
    \begin{bmatrix}
        0 & 0  \\
        1 & 0  \\
    \end{bmatrix}
    \begin{bmatrix}
        0 & 0  \\
        0 & 1  \\
    \end{bmatrix}
    = A
\]

These matrices are linearly independent,  they can produce any matrix in M. 

We can see from \(c_{1}A_1 + \ldots + c_{4}A_4\) that the only way to get 0 is if all coefficients are 0. We can use this basis for many things. \(A_1, A_2, A_4\) are a basis for the upper triangle matrices, \(A_1, A_4\) for diagonal matrices. What about for symmetric matrices? \(A_1, A_4, A_2 + A_3\). 

If list out a few more dimensions:
\begin{enumerate}
    \item The dimension of whole \(n\) by \(n\) matrix space is \(n^2\)
    \item The dimension of the subspace of upper triangular matrices is \(\frac{1}{2}n^2 + \frac{1}{2}n\) 
    \item The dimension of diagonal matrices is \(n\). 
    \item The dimension of the subspace of symmetric matrices is \(\frac{1}{2}n^2 + \frac{1}{2}n\)    
\end{enumerate}

Oh, and remember that \(Z\) is a vector that contains only zero vector. Its basis is a \textbf{empty set}, the one with no vector. NEVER allow zero vector into the basis because then nothing is independent. 

For summary,
\begin{enumerate}
    \item Columns of \(A\) are independent if \(x = 0\) is the only solution to \(Ax = 0\).
    \item \(v_1, \ldots, v_r\) \textbf{span} a space if their combinations fill that space.  
    \item Basis consists of linearly independent vectors that span the space. Ever vector in the space is a unique combination of the basis vector. 
    \item The number of basis vectors per basis is entirely dependent on the space. The number of vectors in a basis is the dimension of the space. 
    \item Pivot columns are one basis for the column space. The dimension of \(C(A)\) is \(r\)     
\end{enumerate}

\section{Dimensions of the Four Subspaces}

\begin{enumerate}
    \item The column space \(C(A)\) and the row space \(C(A^T)\) both have dimension \(r\), the rank of \(A\). 
    \item The nullspace \(N(A)\) has dimension \(n - r\). The left nullspace \(N(A^T)\) has dimension \(m - r\). 
    \item Elimination from \(A\) to \(R_0\) changes \(C(A)\) and \(N(A^T)\) but their dimension doesn't change.
\end{enumerate}

We must separate the difference between rank and dimension. The rank of a matrix counts independent columns, the dimension counts the number of vectors in a basis. We can count both pivots and basis vector. Rank reveals the dimension of the four fundamental subspaces. 

\begin{enumerate}
    \item Row space, \(C(A^T)\), a subspace of \(R^n\), its dimension is \(r\)
    \item Column space, \(C(A)\), a subspace of \(R^m\), its dimension is \(r\)     
    \item Nullspace, \(N(A)\), a subspace of \(R^n\), its dimension is \(n - r\)
    \item Left nullspace, \(N(A^T)\), a subspae of \(R^m\), its dimension is \(m - r\)      
\end{enumerate}

Keep in mind that row space is simply just the column space of \(A^T\). Remember again that the row space and column space have the same dimension \(r\). 

\subsection{The Four Subspaces for \(R_0\)}

We will now try to identify the four subspaces in \(R_0\). 
Though the main point is that \textbf{the four dimensions are the same for \(A\) and \(R_0\)}
\[
    \begin{bmatrix}
        1 & 3 & 5 & 0 & 7  \\
        0 & 0 & 0 & 1 & 2  \\
        0 & 0 & 0 & 0 & 0  \\
    \end{bmatrix}
\]  
We can that see That
\begin{enumerate}
    \item The pivot row is row 1 and 2. 
    \item The pivot column is column 1 and 3. 
    \item \(r = 2\) 
\end{enumerate}

Now, finding the nullspace, \(n - r = 5 - 2 = 3\). We will need to find 3 special solutions. 
\begin{enumerate}
    \item Special solution 1 = \((-3, 1, 0, 0, 0)\) 
    \item Special solution 2 = \((-5, 0, 1, 0, 0)\) 
    \item Special solution 3 = \((-7, 0, 0, -2, 1)\) 
\end{enumerate} 
Now we will find the nullspace of \((R^{T}_{0}\) aka the left nullspace. Its dimension should be \(m - r = 3 - 2 = 1\)  
\[
    (R^{T}_{0}y = 
    \begin{bmatrix}
        1 & 0 & 0  \\
        3 & 0 & 0  \\
        5 & 0 & 0  \\
        0 & 1 & 0  \\
        7 & 2 & 0  \\
    \end{bmatrix}
    \begin{bmatrix}
         y_1 \\
         y_2 \\
         y_3 \\
    \end{bmatrix}
    = 
    \begin{bmatrix}
         0 \\
         0 \\
         0 \\
         0 \\
         0 \\
    \end{bmatrix}
    \text{ is solved by }
    y = 
    \begin{bmatrix}
         0 \\
         0 \\
         y_3 \\
    \end{bmatrix}
\]

Wonder why it's called \textbf{left nullspace}? It's because we can transpose \((R^{T}_{0}y = 0\) to \(y^{T}R_0 = 0^T\), the \(y^T\) is now a row vector to the left of R.

\subsection{The Four Subspaces for A}

A few things to remember,
\begin{enumerate}
    \item \(A\) has the same row space as \(R_0\) and \(R\). Same dimension \(r\) and same basis. Elimination changes the row, but the row space remains untouched. 
    \item The column space of \(A\) has dimension \(r\). The column rank \textbf{equals} the row ranks. The same combination (as in the same column number) are zero or otherwise (meaning both get zero or none get zero) for both \(A\) and \(R_0\).
    \item A has the same nullspace as \(R_0\), same dimension \((n - r)\) and same basis. Elimination doesn't change the solutions to \(Ax = 0\), including the special solutions. 
    \item The left nullspace of \(A\) (the nullspace of \(A^T\)) has dimension \(m-f\). The counting rule for \(A\) was \(r + (n - r) = n\) and the counting rule for \(A^T\) is \(r + (m - r) = m\)        
\end{enumerate}

\textbf{Very IMPORTANT, This is a fundamental theoreom of linear algebra, part 1: The column space and row space both have dimension \(r\). The nullspaces have dimension \(n - r\) and \(m - r\)}

Imagine we have a 11 by 17 matrix with 187 nonzero entries, there are two key facts:
\begin{enumerate}
    \item dimension of \(C(A)\) = dimension of \(C(A^T)\) = rank of \(A\)
    \item dimension of \(C(A)\) + dimension of \(N(A)\) = 17
\end{enumerate}

\textbf{Example 1:}
\[
    A = 
    \begin{bmatrix}
        1 & 2 & 3  \\
        2 & 4 & 6  \\
    \end{bmatrix}
\] 

It has \(m = 2, n = 3, r = 1\). The row space is the line through \((1, 2, 3)\), the nullspace is the plane \(x_1 + 2x_2 + 3x_3 = 0\). The dimension of the row space is 1 since it is a line, and the dimension of the nullspace is 2 since it is a plane. We can do the equation \(r + (n - r)\), where \(n - r\) is the dimension of the nullspace and we get \(1 + 2 = 3\) where \(3 = n\), the number of column. 

We can do the same for left nullspace and column space. The column space is \([1, 2]^T\) since we can see that everything in linearly dependent on it. And nullspace is perpendicular to column space, so try dotting it and see if it gets a 0 or not. In this case, the left nullspace is a line through \([2, 1]^T\)

\textbf{The y's in the left nullspace combine the rows of A to give the zero row,}

\textbf{Example 2:} Time for a more practical question (finally!) Say, we have five equations with four unknowns, one for every nodes. 
\[
    \begin{bmatrix}
        -1 & 1 &  &   \\
        -1 &  & 1 &   \\
         & -1 & 1 &   \\
         & -1 &  & 1  \\
         &  & -1 & 1  \\
    \end{bmatrix}
\]  

To find the nullspace \(N(A)\), set \(b = 0\). The first \(x_1 = x_2\), the second \(x_1 = x_3\), the third \(x_2 = x_3\), the fourth \(x_2 = x_4\), the fifth \(x_3 = x_4\). So all of them are equal, \(x = (c, c, c, c)\). Just one vector is enough to fill the nullspace of \(A\), there it is a line in \(R^4\). We can say \((1, 1, 1, 1)\) is a basis for \(N(A)\). \(N(A)\) has a dim of 1, then \(r\) of \(A\) must be 3 since \(4 - 3 = 1\). We now know the dimension of all four subspaces. 

\textbf{The column space}  \(C(A)\) must be \(r = 3\). We can again easily find out the independent column by rrefing it. 
\[
    R_0 =
    \begin{bmatrix}
        1 & 0 & 0 & -1  \\
        0 & 1 & 0 & -1  \\
        0 & 0 & 1 & -1  \\
        0 & 0 & 0 & 0  \\
        0 & 0 & 0 & 0  \\
    \end{bmatrix}
\]  

\textbf{The left nullspace} \(N(A^T)\), now we can solve \(A^{T}y = 0\). Note that the rows give zero, because of that we can say row 3 = row 2 - row 1, so one of the solution is \(y = (1, -1, 1, 0, 0)\). In case of the lower loop, row 3 = row 4 - row 5, \(y = (0, 0, -1, 1, 1)\). The dimension of \(m - r = 5 - 3 = 2\), giving us the basis for the left nullspace. 

\begin{enumerate}
    \item 1, 2, 3 forms a loop in the graph, (1, 2, 3)are dependent. 
    \item 1, 2, 4 forms a tree in the graph, (1, 2, 4) are independent
\end{enumerate}

So, where the hell did loop and tree come from? Well, they didn't need to happen, we could've used elimination. But it is more elegant. Note that \(Ax = b\) gives voltage at \(x_1, x_2, x_3, x_4\) at the four nodes and \(A^{T}y = 0 \) gives currents \(y_1, y_2, y_3, y_4, y_5\) on the five edges. These wo equations are Kirchhoff's Voltage Law and Kirchhoff's Current Law. Now, I should've said this earlier, but we need a graph for this. I don't know how to make one \textbf{yet}. Check page \textbf{135} in Strang's Linear Algebra. Graphs are the most important model in discrete applied mathematics. 


For summary, \textbf{Incidence matrix A} comes from a connected graph with \(n\) nodes and \(m\) edges. The row space and column space have dimensions of \(r = n - 1\), the nullspaces of \(A\) and \(A^T\) have dimensions 1 and \(m - n + 1\). 
\begin{enumerate}
    \item \(N(A)\) the constant vectors (c,c, \ldots, c) make up the nullspace of A: dim = 1.
    \item \(C(A^T)\)  the edge of any tree give \(r\) independent rows of \(A\): r = n - 1. 
    \item \(C(A)\) Voltage Law: The components of \(Ax\) add to zero around all loops: \(\dim = n - 1\). 
    \item \(N(A^T)\) (Current Law): \(A^{T} y = \text{flow in} - \text{flow out} = 0\), which is solved by loop currents. There are \(m - r = m - n + 1\) independent small loops in the graph.
\end{enumerate}    

\subsection{Rank Two Matrices = Rank One plus Rank One}

This one is a doozy. What it means is then any rank \(r\) matrix can be decomposed into \(r\) amount of rank 1 matrices. 

For example, a rank 2 matrix:
\[
    A = 
    \begin{bmatrix}
        1 & 0 & 3  \\
        1 & 1 & 7  \\
        4 & 2 & 20  \\
    \end{bmatrix}
    = 
    \begin{bmatrix}
        1 & 0  \\
        1 & 1  \\
        4 & 2  \\
    \end{bmatrix}
    \begin{bmatrix}
        1 & 0 & 3  \\
        0 & 1 & 4  \\
    \end{bmatrix}
\]

For reminder, \(C\) is literally just pivot columns and \(R\) is just nonzero-only rref. 

If we put it in letters for columns and rows, we can see \textbf{rank 2 = rank 1 + rank 1}. 
\[
    A = 
    \begin{bmatrix}
        u_1 & u_2 & u_3  \\
    \end{bmatrix}
    \begin{bmatrix}
         v^{T}_1 \\
         v^{T}_2 \\
         \text{zero row} \\
    \end{bmatrix}
    = 
    u_{1}v^{T}_1 + u_{2}v^{T}_2
\] 

To put in Strang's word, \textbf{Columns of \(C\)  times rows of \(R\), every rank \(r\) is a sum of \(r\) rank one matrices.  }

