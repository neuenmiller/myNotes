\chapter{Determinants}
\section{3 by 3 Determinants and Cofactors}

\begin{enumerate}
    \item \(\det \) of \(A= \begin{bmatrix}
        a & b  \\
        c & d  \\
    \end{bmatrix}\) 
    is \(ad - bc\), the singular matrix \(\begin{bmatrix}
        a & 2a  \\
        c & 2c  \\
    \end{bmatrix}\) 
    has \(\det  = 0\)
    \item \(P A = \begin{bmatrix}
        0 & 1  \\
        1 & 0  \\
    \end{bmatrix}
    \begin{bmatrix}
        a & b  \\
        c & d  \\
    \end{bmatrix}
    = 
    \begin{bmatrix}
        c & d  \\
        a & b  \\
    \end{bmatrix}\)  
    has \(\det  P A = bc - ad = - \det A\)  
    \item \(\det \) of \(\begin{bmatrix}
        xa + yA & xb + yB  \\
        c & d  \\
    \end{bmatrix}\) 
    is \(x(ad - bc) + y(Ad - Bc)\), where the \(\det \)  is linear in row 1 by itself. 
    \item \(3 \times 3\)  determinants have \(3! = 6\)terms. 
\end{enumerate}

One thing to remember that determinants reverse sign when two rows are exchanged \((\det A \rightarrow -\det A)\) 

For \(2 \times 2\) matrices, \(\det = 0\) means that \(\frac{a}{c} = \frac{b}{d}\), meaning the columns are parallel. For \(n \times n\), it means that the columns of \(A\) are not independent. 

\subsection{3 by 3 Determinants}

\(3 \times 3\) matrices have 6 terms, starting from \(\det I = 1\). 

\[
\begin{bmatrix}
1 &   &   \\
  & 1 &   \\
  &   & 1
\end{bmatrix}
%
\begin{bmatrix}
  & 1 &   \\
1 &   &   \\
  &   & 1
\end{bmatrix}
%
\begin{bmatrix}
  & 1 &   \\
  &   & 1 \\
1 &   &
\end{bmatrix}
%
\begin{bmatrix}
  &   & 1 \\
  & 1 &   \\
1 &   &
\end{bmatrix}
%
\begin{bmatrix}
  &   & 1 \\
1 &   &   \\
  & 1 &
\end{bmatrix}
%
\begin{bmatrix}
1 &   &   \\
  &   & 1 \\
  & 1 &
\end{bmatrix}
\]

The \(\det \)  are respectively \(+1, -1, +1, -1, +1, -1\). Which means row exchanges multiplies \(\det \) by \(-1\). 

Say we make a matrix of \(\begin{bmatrix}
    a & b & c  \\
    p & q & r  \\
    x & y & z  \\
\end{bmatrix}\) 

It would be 

\[
\begin{bmatrix}
a & & \\
& q & \\
& & z
\end{bmatrix}
%
\begin{bmatrix}
& b & \\
p & & \\
& & z
\end{bmatrix}
%
\begin{bmatrix}
& b & \\
& & r \\
x & &
\end{bmatrix}
%
\begin{bmatrix}
& & c \\
& q & \\
x & &
\end{bmatrix}
%
\begin{bmatrix}
& & c \\
p & & \\
& y &
\end{bmatrix}
%
\begin{bmatrix}
a & & \\
& & r \\
& y &
\end{bmatrix}
\]

We can see that the determinant of \(A\) is linear in each row separately. We can combine them to get \(\det A = aqz + brx + cpy - ary - bpz- cqx\). 

Notice that each term have at least one entry from the row and column. That means that a \(4 \times 4\) matrix will have 24 definitions. 

\[
    A_3 = 
    \begin{bmatrix}
        2 & -1 & 0  \\
        -1 & 2 & -1  \\
        0 & -1 & 2  \\
    \end{bmatrix}
    \text{ \(\det\) = 4 }
\]
\(2 \times 2 \times 2\) already give us \(8\). We'll see how the rest goes. 

\[
\begin{bmatrix}
2 &   &    \\
  &   & -1 \\
  & -1 &
\end{bmatrix}
%
\begin{bmatrix}
  & -1 &   \\
-1 &    &   \\
  &    & 2
\end{bmatrix}
%
\begin{bmatrix}
  & -1 &    \\
  &    & -1 \\
0 &    &
\end{bmatrix}
%
\begin{bmatrix}
  &    & 0  \\
-1 &    &    \\
  & -1 &
\end{bmatrix}
%
\begin{bmatrix}
  &   & 0 \\
  & 2 &   \\
0 &   &
\end{bmatrix}
\]

So, \(8 -2 -2 = 4\). Since we get one number from every row and every column, if any entire row or column are zero, then the whole term is zero. 

\[
A_4=
\begin{bmatrix}
2 & -1 & 0 & 0\\
-1 & 2 & -1 & 0\\
0 & -1 & 2 & -1\\
0 & 0 & -1 & 2
\end{bmatrix},
\qquad
\det A_4
= 2 \det\!\begin{bmatrix}
2 & -1 & 0\\
-1 & 2 & -1\\
0 & -1 & 2
\end{bmatrix}
- (-1)\det\!\begin{bmatrix}
-1 & -1 & 0\\
0 & 2 & -1\\
0 & -1 & 2
\end{bmatrix}.
\]


We can see also see that \(A_4\) is made from smaller matrixes, namely \(A_3\) or \(A_{n-1} \) for general cases. We simply exclude the column and row it is in.  

Now might wonder why the second matrix has a \(-1\)? That is because it's cofactor is  \(a_{12} \)! We can find out what is minus or not by doing \((-1)^{i \times j}\). Here, \(i \times j\) is odd, so the \(-1\) goes through.    


\[
\begin{bmatrix}
+ & - & + & -\\
- & + & - & +\\
+ & - & + & -\\
- & + & - & +
\end{bmatrix}
\]


What does this means?
\[
  \det \begin{bmatrix}
    a & b  \\
    c & d  \\
  \end{bmatrix}
  = \det  
  \begin{bmatrix}
    a &   \\
     & d  \\
  \end{bmatrix}
  + 
  \det  
  \begin{bmatrix}
     & b  \\
    c &   \\
  \end{bmatrix}
  = a \times d \text{\textbf{ MINUS } } b \times c
\]

\subsection{Cofactors and a formula for \(A^{-1} \) }

Cofactor formula of \(3 \times 3\) matrix is simply 
\[
  \det  A = a (qz - ry) + b (rx - pz) + c(py - qx)
\] 
Notice that each cofactor is \(2 \times 2\). Also notice that \(b\) has a cofactor of \(rx - pz\) instead of \(pz - rx\). Why? \(b\) is \(M_{12} \) and \(1 + 2\) is odd. 

\begin{enumerate}
  \item For \(i, j\) in \(C_ij\), remove \(i\) row and \(j\) column from A. 
  \item \(C_{ij}\) equals \(-1^{i+j}\) times the \(\det \) of the remaining minor 
  \item The cofactor formula along row \(i\) is \(\det  A = a_{i1}C_{i1} + \ldots + a_{in}C_{in}    \)  
\end{enumerate}

The cofactor \(C_{ij} \) just collects all term in \(\det A\) that are multiplied by \(a_{ij} \)

\[
  A = 
  \begin{bmatrix}
    a & b  \\
    c & d  \\
  \end{bmatrix}
  , C = 
  \begin{bmatrix}
    d & -c  \\
    -b & a  \\
  \end{bmatrix}
\]

Now, if \(A \times C^T\), we get \(\det A \times I\). 
\[
  AC^T = 
  \begin{bmatrix}
    ad-bc & 0  \\
    0 &  ad-bc \\
  \end{bmatrix}
  = 
  \begin{bmatrix}
    \det A &  0 \\
    0 & \det A  \\
  \end{bmatrix}
  = (\det A) I
\]  
Which gives us:
\[
  \text{Inverse matrix formula }
  \quad 
  A^{-1} = \frac{C^T}{\det A} 
\]

\subsection{Example: The \(-1, 2, -1\) tridiagonal matrix}

Remember, \(D_n = \det A_n = 2 \det(A_{n-1}) + (-1) (1)^{i+j} (M_{12} )\)

\[
\det\!\begin{bmatrix}
2 & -1 & 0 & 0 & \cdots & 0\\
-1 & 2 & -1 & 0 & \cdots & 0\\
0 & -1 & 2 & -1 & \ddots & \vdots\\
0 & 0 & -1 & 2 & \ddots & 0\\
\vdots & \vdots & \ddots & \ddots & \ddots & -1\\
0 & 0 & \cdots & 0 & -1 & 2
\end{bmatrix}
=
2(-1)^{1+1}\!
\det\!\begin{bmatrix}
2 & -1 & 0 & \cdots & 0\\
-1 & 2 & -1 & \ddots & \vdots\\
0 & -1 & 2 & \ddots & 0\\
\vdots & \ddots & \ddots & \ddots & -1\\
0 & \cdots & 0 & -1 & 2
\end{bmatrix}
\;-\;
1\,(-1)^{1+2}\!
\det\!\begin{bmatrix}
-1 & -1 & 0 & \cdots & 0\\
0 & 2 & -1 & \ddots & \vdots\\
0 & -1 & 2 & \ddots & 0\\
\vdots & \ddots & \ddots & \ddots & -1\\
0 & \cdots & 0 & -1 & 2
\end{bmatrix}.
\]


The first one is plain old \(n - 1\), but the second one, remember that it is \(M_{12} \). And the cofactor is \((-1)^{1+2} = -1\). 
\[
  \det  A_n = 2 \det A_{n-1} - \det A_{n-2}  
\]   

Working it, we find that \(A_n = 2, 3, 4, 5\) for \(n = 1, 2, 3, 4\). We now see that \(\det A_n = n + 1\) for every \(n\). Cofactor formula is most useful when the matrix is mostly zero, so we have few cofactor to find.         

\section{Computing and Using Determinants}

\begin{enumerate}
  \item \(\det A^T = \det A, \det AB = (\det A)(\det B), \vert \det Q \vert = 1 \).
  \item Elimination matrix have \(\det E = 1\) therefore \(\det EA = \det A\).  
  \item Cramer's rule find \(x = A^{-1} b \) from ratio of determinant, slow. 
  \item \(\det A = \pm \) product of the pivots in \(A = LU\), fast. 
  \item Big formula for \(\det A\) has \(n!\) term from \(n!\) permutation, very slow.      
\end{enumerate}

Determinant of a square matrix tells us that invertible matrix has \(\det A \neq 0\), where a singular matrix can have \(\det A = 0\). 

\[
\det\!\begin{bmatrix}
a & b & c\\
0 & q & r\\
0 & 0 & z
\end{bmatrix}
=
\det\!\begin{bmatrix}
a & 0 & 0\\
0 & q & 0\\
0 & 0 & z
\end{bmatrix}
= aqz.
\]

It tells us a few things. 

\begin{enumerate}
  \item \(\det A^T = \det A\)
  \item \(\det (AB) = (\det A)(\det B)\)  
  \item Orthogonal matrices \(Q\) have determinant \(\pm 1\).  
\end{enumerate}

The last one requires a bit more explanation. 
\[
  Q^T Q = I, (\det Q)^2 = (\det Q^T)(\det Q) = 1, \text{ therefore } Q = \pm 1. 
\]

\emph{Invertible matrices have \(\det A = \pm\) product of the pivots. }

If \(A = LU\), \(\det A = (\det L)(\det U) = \det U\). 
If \(PA = LU\), then there was row exchanges/permutation and \(P = \pm 1\).     

\subsection{Proving the Properties}

We must agree on 3 things.
\begin{enumerate}
  \item \(\det I = 1\) 
  \item Exchanging two rows of \(A\) turns \(\det A \rightarrow - \det A\)
  \item If row 1 of \(A\) is a combination \(cv + dw\) then add 2 determinants.     
\end{enumerate}

\[
\det\begin{bmatrix}
c\,v + d\,w\\
\text{row }2\\
\vdots\\
\text{row }n
\end{bmatrix}
= c\,\det\!\begin{bmatrix}
v\\
\text{row }2\\
\vdots\\
\text{row }n
\end{bmatrix}
+ d\,\det\!\begin{bmatrix}
w\\
\text{row }2\\
\vdots\\
\text{row }n
\end{bmatrix}.
\]

\(\det A\) is linear with respect to every row separately. If \(A\) has two equal rows, its determinant is \textbf{zero}. If you try to swap two rows, that mean it swaps sign, but if it has to swap sign and stay the same, that means it's 0. Rule 3 means that subtracting \(d\) times row \(i\) from row \(j\) leaves \(\det A\) unchanged. 

\[
\det\!\begin{bmatrix}
\text{row }1\\
\text{row }2 - d\,\text{row }1\\
\vdots\\
\text{row }n
\end{bmatrix}
=
\det\!\begin{bmatrix}
\text{row }1\\
\text{row }2\\
\vdots\\
\text{row }n
\end{bmatrix}
-
d\,\det\!\begin{bmatrix}
\text{row }1\\
\text{row }1\\
\vdots\\
\text{row }n
\end{bmatrix}
= \det A.
\]

This is linearity in row 2 with row 1 fixed.  It means elimination steps from the original matrix \(A\) to an upper triangular \(U\) do not change the determinant. Elimination is the way to simplify \(A\) and its determinant. 

\[
  \det A = \det U = U_{11} U_{22} U_{33}  \ldots U_{nn} 
\]   

\subsection{Cramer's Rule to Solve \(Ax = b\) }

Cramer's rule work by letting \(M_1\) has determinant \(x_1\). If you multiply it by \(A\), the first column becomes \(Ax\) which is \(b\). 

\[
A
\begin{bmatrix}
x_1 & 0 & 0\\
x_2 & 1 & 0\\
x_3 & 0 & 1
\end{bmatrix}
=
\begin{bmatrix}
b_1 & a_{12} & a_{13}\\
b_2 & a_{22} & a_{23}\\
b_3 & a_{32} & a_{33}
\end{bmatrix}
= B_1.
\]

Which gives us the product rule:
\[
  (\det A)(x_1) = \det B_1 \text{ or }
  x_1 = \frac{\det B_1}{\det A}
\]

\[
\begin{bmatrix} a_1 & a_2 & a_3 \end{bmatrix}
\begin{bmatrix}
1 & x_1 & 0\\
0 & x_2 & 0\\
0 & x_3 & 1
\end{bmatrix}
=
\begin{bmatrix} a_1 & b & a_3 \end{bmatrix}
= B_2.
\]

Same thing but \((\det A)(x_2) = \det B_2\) or \(x_2 = \frac{\det B_2}{\det A}\)  

\textbf{Example 1:} Solve for \(3x_1 + 4x_2 = 2, 5x_1 + 6x_2 = 4\)

\[
  \det A = 
  \begin{bmatrix}
    3 & 4  \\
    5 & 6  \\
  \end{bmatrix}
  \quad 
  \det B_1 = 
  \begin{bmatrix}
    2 & 4  \\
    4 & 6  \\
  \end{bmatrix}
  \quad 
  \det B_2 = 
  \begin{bmatrix}
    3 & 2  \\
    5 & 4  \\
  \end{bmatrix}
\]

Determinant of \(B_1\) and \(B_2\) are \(-4\) and \(2\), divided by \(\det A = -2\)



\[
  x_1 = \frac{-4}{-2} = 2
  \quad 
  x_2 = \frac{2}{-2} = -1
  \quad 
  \text{Check }
  \begin{bmatrix}
    3 & 4  \\
    5 & 6  \\
  \end{bmatrix}
  \begin{bmatrix}
     2 \\
     -1 \\
  \end{bmatrix}
  = 
  \begin{bmatrix}
     2 \\
     4 \\
  \end{bmatrix}
\]


And so the Cramer's rule is:
\[
  x_1 = \frac{\det B_1}{\det A}
  \quad 
  x_2 = \frac{\det B_2}{\det A}
  \quad 
  \ldots 
  \quad 
  x_n = \frac{\det B_n}{\det A}
\]



\textbf{Example 2:} We want to find a \(A[x \quad y] = I\). 
\[
  Ax = 
  \begin{bmatrix}
    a & b  \\
    c & d  \\
  \end{bmatrix}
  \begin{bmatrix}
     x_1 \\
     x_2 \\
  \end{bmatrix}
  = 
  \begin{bmatrix}
     1 \\
     0 \\
  \end{bmatrix}
  \quad 
  Ay = 
  \begin{bmatrix}
    a & b  \\
    c & d  \\
  \end{bmatrix}
  \begin{bmatrix}
     y_1 \\
     y_2 \\
  \end{bmatrix}
  = 
  \begin{bmatrix}
     0 \\
     1 \\
  \end{bmatrix}
\]



We will need four determinant for \(x_1, x_2, y_1, y_2\)



Giving us the determinants: \(d, -c, -b, a\). 


\[
  x_1 = \frac{d}{\vert A \vert },
  \quad 
  x_2 = \frac{-c}{\vert A \vert },
  \quad 
  x_3 = \frac{-b}{\vert A \vert }
  \quad 
  x_4 = \frac{a}{\vert A \vert }
  \text{ and }
  A^{-1} = \frac{1}{ad - bc}
  \begin{bmatrix}
    d & -b  \\
    -c & a  \\
  \end{bmatrix} 
\]

\(A^{-1} \) involves the cofactor of \(A\), and the determinant in each \(B_j\) in Cramer's rule is a cofactor of \(A\). 

\[
\det\!\begin{bmatrix}
1 & a_{12} & a_{13}\\
0 & a_{22} & a_{23}\\
0 & a_{32} & a_{33}
\end{bmatrix}\quad
\det\!\begin{bmatrix}
a_{11} & 1 & a_{13}\\
a_{21} & 0 & a_{23}\\
a_{31} & 0 & a_{33}
\end{bmatrix}\quad
\det\!\begin{bmatrix}
a_{11} & a_{12} & 1\\
a_{21} & a_{22} & 0\\
a_{31} & a_{32} & 0
\end{bmatrix}
\]

\[
  C_{ij} = B_j 
\]

\(B_1\) is the cofactor \(C_{11} = a_{22}a_{33} - a_{23}a_{32}\), \(\det B_2 = -(a_{21}a_{33} - a_{23}a_{31})\). Notice how \(C_{12} \) go into column 1 of \(A^{-1} \)? 
\[
  (A^{-1} )_{ij} = \frac{C_{ji} }{\det A} 
  \quad 
  A^{-1} = \frac{C^T}{\det A} 
\]     

\subsection{The Big Formula for the Determinant: \(n!\) term }



\section{Areas and Volumes by Determinants}

\begin{enumerate}
  \item Parallelogram in \(2D = \text{ side } e_1 = (a, b), e_2 = (c, d)\). 
  \item Area of parallelogram = \(\vert \det E = [e_1, e_2] \vert = \vert ad - bc \vert  \)  
  \item Tilted box in \(3D\) starts with three edges \(e_1, e_2, e_3\) out from \(0, 0, 0\). 
  \item Volume of a tilted box \(= \vert 3 \times 3 \text{ edge matrix } E  \vert \).    
\end{enumerate}

How we can find the area of a parallelogram? Parallelogram can be broken down into 2 triangle, the area is simply \(base \times height\). However, we are not given base and height, but the position of the corners. Supposed the corners are \((0, 0), (a, b), (c, d)\), the forth corner would be \((a+c, b+d)\). 

The equation would simply be: 
\[
  \text{Area of parallelogram} = \vert \det  \vert = 
  \pm 
  \begin{bmatrix}
    a & c  \\
    b & d  \\
  \end{bmatrix}
  = \vert ad - bc \vert 
\]

That's nice for \(2D\), but we will need to move to \(3D\). Start with four corners, \((0, 0, 0), (a, b, c), (p, q, r), (x, y, z)\). 

\[
  \text{Volume of box} = \vert \det  \vert = \pm 
  \begin{bmatrix}
    a & p & x  \\
    b & q & y  \\
    c & r & z  \\
  \end{bmatrix} 
\]

\subsection{Areas and Volumes by Linear Algebra}

\emph{A box in \(n\) dimensions has \(n\) edges \(e_1, e_2, \ldots, e_n\) going out from the origin  }. Parallel in \(2D\) have \(e_1, e_2\) coming out of them. 

\[
  \text{ Edge matrix} \quad 
  E_2 = 
  \begin{bmatrix}
    a & c  \\
    b & d  \\
  \end{bmatrix}, \quad 
  \begin{bmatrix}
    e_1 & \ldots & e_n  \\
  \end{bmatrix} \quad 
  Columns = Box edges
\]

There are ways to prove that volume of the box is \(\det E\): 
\begin{enumerate}
  \item Lower times upper triangle, \(E = LU\). 
  \item Orthogonal times upper triangular R, \(E = QR\).
  \item Orthogonal-Diagonal-Orthogonal, \(E = U \Sigma V^T\).    
\end{enumerate} 

Remember, \(\det L = 1, \det Q = \pm 1\). \(\det L = \vert \det Q \vert = \vert \det U \vert = \vert \det V \vert = 1  \). 

\[
  \text{ Box volume } = \vert \det E \vert = \vert \det U \vert = \vert \det R \vert = \vert \det \Sigma  \vert   
\]

Do remember that \emph{multiplying all points by an orthogonal matrix \(Q\) does not change the volume.  Length and angles and box shapes and volumes are not changed by rotations \(Q\) }. Same goes for curved region, where they are cut into really small cubes plus super thin curved pieces which approach zero. 

\(R\) is a triangular matrix, its box has a volume we can compute. For parallelogram, \(R = \begin{bmatrix}
  u & v  \\
  0 & w  \\
\end{bmatrix}\)  has the base of \(u\) and the diagonal entries of \(u\) and \(w\), the volume is simply \(\det R\).    