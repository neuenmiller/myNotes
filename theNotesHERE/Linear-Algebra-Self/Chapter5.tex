\chapter{Determinants}
\section{3 by 3 Determinants and Cofactors}

\begin{enumerate}
    \item \(\det \) of \(A= \begin{bmatrix}
        a & b  \\
        c & d  \\
    \end{bmatrix}\) 
    is \(ad - bc\), the singular matrix \(\begin{bmatrix}
        a & 2a  \\
        c & 2c  \\
    \end{bmatrix}\) 
    has \(\det  = 0\)
    \item \(P A = \begin{bmatrix}
        0 & 1  \\
        1 & 0  \\
    \end{bmatrix}
    \begin{bmatrix}
        a & b  \\
        c & d  \\
    \end{bmatrix}
    = 
    \begin{bmatrix}
        c & d  \\
        a & b  \\
    \end{bmatrix}\)  
    has \(\det  P A = bc - ad = - \det A\)  
    \item \(\det \) of \(\begin{bmatrix}
        xa + yA & xb + yB  \\
        c & d  \\
    \end{bmatrix}\) 
    is \(x(ad - bc) + y(Ad - Bc)\), where the \(\det \)  is linear in row 1 by itself. 
    \item \(3 \times 3\)  determinants have \(3! = 6\)terms. 
\end{enumerate}

One thing to remember that determinants reverse sign when two rows are exchanged \((\det A \rightarrow -\det A)\) 

For \(2 \times 2\) matrices, \(\det = 0\) means that \(\frac{a}{c} = \frac{b}{d}\), meaning the columns are parallel. For \(n \times n\), it means that the columns of \(A\) are not independent. 

\subsection{3 by 3 Determinants}

\(3 \times 3\) matrices have 6 terms, starting from \(\det I = 1\). 

\[
\begin{bmatrix}
1 &   &   \\
  & 1 &   \\
  &   & 1
\end{bmatrix}
%
\begin{bmatrix}
  & 1 &   \\
1 &   &   \\
  &   & 1
\end{bmatrix}
%
\begin{bmatrix}
  & 1 &   \\
  &   & 1 \\
1 &   &
\end{bmatrix}
%
\begin{bmatrix}
  &   & 1 \\
  & 1 &   \\
1 &   &
\end{bmatrix}
%
\begin{bmatrix}
  &   & 1 \\
1 &   &   \\
  & 1 &
\end{bmatrix}
%
\begin{bmatrix}
1 &   &   \\
  &   & 1 \\
  & 1 &
\end{bmatrix}
\]

The \(\det \)  are respectively \(+1, -1, +1, -1, +1, -1\). Which means row exchanges multiplies \(\det \) by \(-1\). 

Say we make a matrix of \(\begin{bmatrix}
    a & b & c  \\
    p & q & r  \\
    x & y & z  \\
\end{bmatrix}\) 

It would be 

\[
\begin{bmatrix}
a & & \\
& q & \\
& & z
\end{bmatrix}
%
\begin{bmatrix}
& b & \\
p & & \\
& & z
\end{bmatrix}
%
\begin{bmatrix}
& b & \\
& & r \\
x & &
\end{bmatrix}
%
\begin{bmatrix}
& & c \\
& q & \\
x & &
\end{bmatrix}
%
\begin{bmatrix}
& & c \\
p & & \\
& y &
\end{bmatrix}
%
\begin{bmatrix}
a & & \\
& & r \\
& y &
\end{bmatrix}
\]

We can see that the determinant of \(A\) is linear in each row separately. We can combine them to get \(\det A = aqz + brx + cpy - ary - bpz- cqx\). 

Notice that each term have at least one entry from the row and column. That means that a \(4 \times 4\) matrix will have 24 definitions. 

\[
    A_3 = 
    \begin{bmatrix}
        2 & -1 & 0  \\
        -1 & 2 & -1  \\
        0 & -1 & 2  \\
    \end{bmatrix}
    \text{ \(\det\) = 4 }
\]
\(2 \times 2 \times 2\) already give us \(8\). We'll see how the rest goes. 

\[
\begin{bmatrix}
2 &   &    \\
  &   & -1 \\
  & -1 &
\end{bmatrix}
%
\begin{bmatrix}
  & -1 &   \\
-1 &    &   \\
  &    & 2
\end{bmatrix}
%
\begin{bmatrix}
  & -1 &    \\
  &    & -1 \\
0 &    &
\end{bmatrix}
%
\begin{bmatrix}
  &    & 0  \\
-1 &    &    \\
  & -1 &
\end{bmatrix}
%
\begin{bmatrix}
  &   & 0 \\
  & 2 &   \\
0 &   &
\end{bmatrix}
\]

So, \(8 -2 -2 = 4\). Since we get one number from every row and every column, if any entire row or column are zero, then the whole term is zero. 

\[
A_4=
\begin{bmatrix}
2 & -1 & 0 & 0\\
-1 & 2 & -1 & 0\\
0 & -1 & 2 & -1\\
0 & 0 & -1 & 2
\end{bmatrix},
\qquad
\det A_4
= 2 \det\!\begin{bmatrix}
2 & -1 & 0\\
-1 & 2 & -1\\
0 & -1 & 2
\end{bmatrix}
- (-1)\det\!\begin{bmatrix}
-1 & -1 & 0\\
0 & 2 & -1\\
0 & -1 & 2
\end{bmatrix}.
\]


We can see also see that \(A_4\) is made from smaller matrixes, namely \(A_3\) or \(A_{n-1} \) for general cases. We simply exclude the column and row it is in.  

Now might wonder why the second matrix has a \(-1\)? That is because it's cofactor is  \(a_{12} \)! We can find out what is minus or not by doing \((-1)^{i \times j}\). Here, \(i \times j\) is odd, so the \(-1\) goes through.    


\[
\begin{bmatrix}
+ & - & + & -\\
- & + & - & +\\
+ & - & + & -\\
- & + & - & +
\end{bmatrix}
\]


What does this means?
\[
  \det \begin{bmatrix}
    a & b  \\
    c & d  \\
  \end{bmatrix}
  = \det  
  \begin{bmatrix}
    a &   \\
     & d  \\
  \end{bmatrix}
  + 
  \det  
  \begin{bmatrix}
     & b  \\
    c &   \\
  \end{bmatrix}
  = a \times d \text{\textbf{ MINUS } } b \times c
\]

\subsection{Cofactors and a formula for \(A^{-1} \) }

Cofactor formula of \(3 \times 3\) matrix is simply 
\[
  \det  A = a (qz - ry) + b (rx - pz) + c(py - qx)
\] 
Notice that each cofactor is \(2 \times 2\). Also notice that \(b\) has a cofactor of \(rx - pz\) instead of \(pz - rx\). Why? \(b\) is \(M_{12} \) and \(1 + 2\) is odd. 

\begin{enumerate}
  \item For \(i, j\) in \(C_ij\), remove \(i\) row and \(j\) column from A. 
  \item \(C_{ij}\) equals \(-1^{i+j}\) times the \(\det \) of the remaining minor 
  \item The cofactor formula along row \(i\) is \(\det  A = a_{i1}C_{i1} + \ldots + a_{in}C_{in}    \)  
\end{enumerate}

The cofactor \(C_{ij} \) just collects all term in \(\det A\) that are multiplied by \(a_{ij} \)

\[
  A = 
  \begin{bmatrix}
    a & b  \\
    c & d  \\
  \end{bmatrix}
  , C = 
  \begin{bmatrix}
    d & -c  \\
    -b & a  \\
  \end{bmatrix}
\]

Now, if \(A \times C^T\), we get \(\det A \times I\). 
\[
  AC^T = 
  \begin{bmatrix}
    ad-bc & 0  \\
    0 &  ad-bc \\
  \end{bmatrix}
  = 
  \begin{bmatrix}
    \det A &  0 \\
    0 & \det A  \\
  \end{bmatrix}
  = (\det A) I
\]  
Which gives us:
\[
  \text{Inverse matrix formula }
  \quad 
  A^{-1} = \frac{C^T}{\det A} 
\]

\subsection{Example: The \(-1, 2, -1\) tridiagonal matrix}

Remember, \(D_n = \det A_n = 2 \det(A_{n-1}) + (-1) (1)^{i+j} (M_{12} )\)

\[
\det\!\begin{bmatrix}
2 & -1 & 0 & 0 & \cdots & 0\\
-1 & 2 & -1 & 0 & \cdots & 0\\
0 & -1 & 2 & -1 & \ddots & \vdots\\
0 & 0 & -1 & 2 & \ddots & 0\\
\vdots & \vdots & \ddots & \ddots & \ddots & -1\\
0 & 0 & \cdots & 0 & -1 & 2
\end{bmatrix}
=
2(-1)^{1+1}\!
\det\!\begin{bmatrix}
2 & -1 & 0 & \cdots & 0\\
-1 & 2 & -1 & \ddots & \vdots\\
0 & -1 & 2 & \ddots & 0\\
\vdots & \ddots & \ddots & \ddots & -1\\
0 & \cdots & 0 & -1 & 2
\end{bmatrix}
\;-\;
1\,(-1)^{1+2}\!
\det\!\begin{bmatrix}
-1 & -1 & 0 & \cdots & 0\\
0 & 2 & -1 & \ddots & \vdots\\
0 & -1 & 2 & \ddots & 0\\
\vdots & \ddots & \ddots & \ddots & -1\\
0 & \cdots & 0 & -1 & 2
\end{bmatrix}.
\]


The first one is plain old \(n - 1\), but the second one, remember that it is \(M_{12} \). And the cofactor is \((-1)^{1+2} = -1\). 
\[
  \det  A_n = 2 \det A_{n-1} - \det A_{n-2}  
\]   

Working it, we find that \(A_n = 2, 3, 4, 5\) for \(n = 1, 2, 3, 4\). We now see that \(\det A_n = n + 1\) for every \(n\). Cofactor formula is most useful when the matrix is mostly zero, so we have few cofactor to find.     

\section{Computing and Using Determinants}



\section{Areas and Volumes by Determinants}


