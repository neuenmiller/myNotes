\chapter{Orthogonality}

Two vectors are orthogonal (translated as right-angled from Greek) when their dot product is zero: \(v \cdot w = v^{T}w = 0\). The vectors in the two subspaces, the vectors in a basis, the column vectors in Q. Orthogonal vectors also have a curious behavior where it is also like pythagoras's theoreom. 
\[
    v^{T}w = 0 \text{ and } \left\lVert v \right\rVert^2 + \left\lVert w \right\rVert^2 = \left\lVert v + w \right\lVert^2
\] 

Important part, \textbf{the fundamental subspaces are orthogonal}
\begin{enumerate}
    \item N(A) contains all vectors orthogonal to the row space \(C(A^T)\).
    \item \(N(A^T)\)  contains all vectors orthogonal to the column space \(C(A)\). 
\end{enumerate} 
\(Ax = 0\) makes \(x\) orthogonal to each row, \(A^{T}y = 0\) make \(y\) orthogonal to each column. 

A key idea in this chapter is \textbf{projection}: If \(b\) is outside the column space of \(A\), then we must find the closest point \(p\) that is still inside. The line from \(b\) to \(p\) shows the error \(e\), and that line is perpendicular to the column space.    THe \textbf{least squares equation} \(A^{T}Ax = A^{T}b\) produces the closest \(p = Ax\) and smallest possible \(e\), it also gives the best \(e\) when \(Ax = b\) is unsolvable. The best \(x\) makes \(\left\lVert Ax - b \right\lVert\) as small as possible, the least squares. \(A^{T}Ax = A^{T}b\) is easy when \(A^{T}A = I\). Then \(A\) has orthonormal columns perpendicular unit vector. Remember that \(Q\) has \(Q^{T}Q = I\) and \(QR = A\), where the R is upper triangle. Orthogonal matrices are perfect for computations, \(A = QR\) is even better than \(A+LU\)               

\section{Orthogonality of Vectors and Subspaces}

\begin{enumerate}
    \item Orthogonal vectors have \(v^{T}w = 0\), then \(\left\lVert v \right\lVert^2\) + \(\left\lVert w \right\lVert\) = \(\left\lVert v + w \right\lVert^2\) as in \(a^2 + b^2 = c^2\)
    \item Subspace \(V\) and \(W\) are orthogonal when \(v^{T}w = 0\) for every \(v\) in \(V\) and every \(w\) in \(W\). 
    \item Row space of \(A\) is orthogonal to nullspace, column space of \(A\) is orthogonal to left nullspace.    
    \item The dimensions add to \(r + (n - r) = n\) and \(r + (m - r) = m\): orthogonal complements. 
    \item If \(n\) vectors in \(R^n\) are independent, they span \(R^n\). If \(n\) vectors span \(R^n\), they are independent.                 
\end{enumerate}

How can we proof that the entirety of the nullspace of \(A\) is orthogonal to the row space of \(A\)? Look at \(Ax = 0\). 
\[
    Ax = 
    \begin{bmatrix}
         \text{row 1 of \(A\) } \\
         \vdots \\
         \text{row $m$ of $A$} \\
    \end{bmatrix}
    \begin{bmatrix}
         x \\
    \end{bmatrix}
    =
    \begin{bmatrix}
         0 \\
         \vdots \\
         0 \\
    \end{bmatrix}
\]  
Notice how they are all zero? That means all of the row space is orthogonal to the nullspace. 

But given that we like to work with columns, another way you can proof it is: \( x^{T}(A^{T}y) = (Ax)^{T}y = 0^{T}y = 0 \)

\textbf{Importantly}, column space \(C(A)\) and left nullspace \(N(A^T)\) is also a perpendicular pair. The proof is more or less the same except we use \(A^T\) instead of \(A\). 

\textbf{Example 1:} The two rows of \(A\) are perpendicular to \(x\) in the nullspace of \(A\): 
\[
    Ax = 
    \begin{bmatrix}
        1 & -2 & 1  \\
        1 & 0 & -1  \\
    \end{bmatrix}
    \begin{bmatrix}
         1 \\
         1 \\
         1 \\
    \end{bmatrix}
    = 
    \begin{bmatrix}
         0 \\
         0 \\
    \end{bmatrix}
\]
or
\[
    A^{T}y = 
    \begin{bmatrix}
        1 & 1  \\
        -2 & 0 \\
        1 & -1  \\
    \end{bmatrix}
    \begin{bmatrix}
         0 \\
         0 \\
    \end{bmatrix}
    = 
    \begin{bmatrix}
         0 \\
         0 \\
         0 \\
    \end{bmatrix}
\]   

This is a fairly extreme case since \(C(A)\) is all of  \(R^2\) and the nullspace of \(A^T\) is a zero vector. The total dimension is \(2 + 1 = 3\). The subspaces accounted for all vectors in \(R^3 = R^n\). The column space and left nullspace have a dimension of \(2 + 0 = 2\), accounting for all vectors in \(R^{2} = R^m\). 

Note that \textbf{if V and W are orthogonal subspaces in \(R^{n} \) then \(\dim V + \dim W \leq n\)    }
Of course, imagine our house. We got a wall and a floor, they can't be orthogonal subspaces since they are both \(R^2\) and our world is \(R^3\), \(2 + 2 \geq 3\). So this is wrong. Some vector will lie in both the wall and the floor, which is the line where the wall meets the floor in both subspaces. 

Two orthogonal subspaces that account for the whole space have a special name, \textbf{orthogonal complements}. Orthogonal complement of \(V^{\perp}\) of \(V\) contains all vectors orthogonal to \(V\).    
So the two pairs of subspace in linear algebra are actually orthogonal complements. 
\begin{enumerate}
    \item Row space and null space, \(r + ( n - r) = n\)
    \item Column space and left nullspace \(r + (m - r)= m\)  
\end{enumerate}

Any vector \(x\) in \(R^n\) is the sum \(x = x_{\text{row}} + x_{\text{null}}\) of its row space and component and its null space component. Same goes for \(y\) in \(R^m\) is the sum \(y = y_{\text{col}} + y_{\text{null}}\), between its column space component and its component in \(N(A^T)\).

\textbf{Fundamental theoreom of Linear Algebra, Part 2:
\begin{enumerate}
    \item \(N(A)\) is the orthogonal complement of the row space \(C(A^T)\)  in \(R^n\)
    \item \(N(A^T)\) is the orthogonal complement of the column space \(C(A)\) in \(R^m\)    
\end{enumerate}} 


Check the figure on page 146 of Strang's LA. It will us that the complete solution to \(Ax = b\) is \(x = \text{ one } x_r + \text{ any } x_n \). Then the minimum norm solution to \(Ax = b\) is \(x = x_r\) from the row space plus \(x_n = 0\) from the nullspace, from \(\lVert x \rVert^2 = \lVert x_r \rVert^2 + \lVert  x_n \rVert^2   \)     

Every vector \(Ax\) is in the column space. Multiplying by \(A\) cannot do anything else. More importantly, every \(b\) in the column space comes from exactly one vector \(x_r\) in the row space.  

\textbf{Example 2: Every matrix of rank \(r\) has an \(r\) by \(r\) invertible submatrix. A has rank = 2:   }
\[
    \begin{bmatrix}
        1 & 2 & 3 & 4 & 5   \\
        1 & 2 & 4 & 5 & 6  \\
        1 & 2 & 4 & 5 & 6  \\
    \end{bmatrix}
    \text{ contains }
    \begin{bmatrix}
        1 & 3  \\
        1 & 4  \\
    \end{bmatrix}
    \text{ in the pivot rows and pivot columns.}
\] 

So if the submatrix \(C\) has \(r\) independent columns, then \(C\) and \(C^T\) has \(r\) independent columns. This locates an \(r\) by \(r\) invertible submatrix of \(A\). 

\subsection{Combing Bases from Subspaces}

Basis have two properties:
\begin{enumerate}
    \item They are linearly independent 
    \item They span the space
\end{enumerate}

However, the two properties implies eachother. \textbf{If there are \(n\) columns of independent vector in \(A\), then they span \(R^n\), \(Ax=b is solvable\). If \(n\) vectors span \(R^n\), they must be independent; \(Ax=b\) has one solution. If \(AB = I\) for square matrix, then \(BA = I\) too.   } 
We can also start from the opposite side, say \(Ax=b\) can be solved for every \(b\), forcing \textbf{existence of solution}. This means elimination produced no zero rows. There are \(n\) and not free variables. The nullspace contains only \(x=0\), ending in \textbf{uniqueness of solutions}. 

\textbf{Example 3:}
\[
    \text{For } A = 
    \begin{bmatrix}
        1 & 2  \\
        3 & 6  \\
    \end{bmatrix}
    \text{ split } x = 
    \begin{bmatrix}
         4 \\
         3 \\
    \end{bmatrix}
    \text{ into } x_r + x_n = 
    \begin{bmatrix}
         2 \\
         4 \\
    \end{bmatrix}
    + 
    \begin{bmatrix}
         2 \\
         -1 \\
    \end{bmatrix}
\] 

The vector (2, 4) is in the row space, while the orthogonal vector is from the nullspace (2, -1). 

\textbf{Example 4: Suppose \(S\) a six dimensional subspace of nine-dimensional space \(R^9\) }

\begin{enumerate}
    \item What are the possible dimensions of subspaces orthogonal to S? \textbf{0, 1, 2, 3 since \(S\) already took 6 of the total 9.}
    \item What are the possible dimension of the orthogonal complement \(S^{\perp}\) of \(S\)? \textbf{3, because they are asking for THE orthogonal complement, the one that contains everything else. So the biggest one, 3, is the answer. }
    \item What is the smallest size of matrix \(A\) that has row space \(S\)? \textbf{6 by 9, because A is a matrix of \(m \times n\) and given  \(R^n = R^9\), \(m \times 9\). For m, the dimension of the row space is the rank of m, so m = 6. Therefore, \(6 \ times 9\)  }
    \item What is the smallest possible size of a matrix \(B\) that has nullspace \(S^{\perp}\)? \textbf{6 by 9. We are told that \(R^9\) have an \(n\) of 9. And that \(S^{\perp} = 3\) from the second question, so we know that \(r + null = n \), we got \(r + 3 = 9\), r = 6. \(6 \times 9\)      }         
\end{enumerate} 

\section{Projections onto Lines and Subspaces}

\section{Least Squares Approximations}

\section{Orthogonal Matrices and Gram-Schmidt}

\section{The Pseudoinverse of a Matrix}
