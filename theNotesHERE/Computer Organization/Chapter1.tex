\chapter{Introduction to CS2115}

\section{Lecturers}
Taught by professor Nan Guan, email: \emph{naguan@cityu.edu.hk}. Send question to the email, resend it in 2 days if no response are given. Email are preferred, even if canvas route are available. 

\section{Course delivery}

Course will be done face-to-face and with tutorial, remember to check canvas for info and files. 

\section{Schedule}


\begin{tabular}{@{}lcc@{}}
\toprule
\textbf{Week} & \textbf{Lecture} & \textbf{Tutorial}\\
\midrule
1  & $\checkmark$ & \\
2  & $\checkmark$ & \\
3  & $\checkmark$ & \\
4  & $\checkmark$ & $\checkmark$\\
5  & $\checkmark$ & $\checkmark$\\
6  & $\checkmark$ & $\checkmark$\\
7  & $\checkmark$ & $\checkmark$\\
8  & $\checkmark$ \ \colorbox{yellow}{\textbf{(Midterm)}} & $\checkmark$\\
9  & $\checkmark$ & $\checkmark$\\
10 & $\checkmark$ & $\checkmark$\\
11 & $\checkmark$ & $\checkmark$\\
12 & $\checkmark$ & \\
13 & $\checkmark$ & \\
\bottomrule
\end{tabular}


\section{Assessment}
\begin{enumerate}
    \item Closed-book, final exam will account for 70\%
    \item Continuous Assessment will count for 30\%, with the midterm account for 10\%. Homework will account for 20\%, with their being four homeworks.
\end{enumerate}

\begin{tabular}{@{}lcc@{}}
\toprule
\textbf{Homework} & \textbf{Tentative Due date} & \textbf{Weight}\\
\midrule
HW 1 & Sep.\ 28 & 2\%\\
HW 2 & Oct.\ 19 & 5\%\\
HW 3 & Nov.\ 9  & 5\%\\
HW 4 & Dec.\ 7  & 8\%\\
\bottomrule
\end{tabular}

\emph{Homework submissions are held to the same standard regardless of quantity of student.}

\section{Topics}

\begin{enumerate}
    \item Number Systems
    \item Circuits 
    \item ISA \& Assembly 
    \item CPU 
    \item Memory 
    \item I/O \& Storage
    \item GPU \& Accelerators 
    \item Performance Analysis
    \item Other more advanced topics
\end{enumerate}

\section{Software needed}
\begin{enumerate}
    \item Logisim
    \item MARS (\emph{NEED JAVA RUNTIME ENVIRONMENT})
\end{enumerate}

\section{Textbook and Readings}

Computer Organization and Embedded Systems 

\section{What is this course about?}

This course is about how computer works. 

\section{Why do we learn this course?}

\begin{enumerate}
    \item Computer scientists should understand computer, even though most don't. 
    \item Knowing how computer works allows for squeezing all the performance out of a chip.
    \item Knowing the computer will allow us to pick the right for the job. 
\end{enumerate}

\section{Quick openers}

\subsection{Turing Machine}

There have been a long-standing question whether all questions can be solved by a machine or not, Turing thinks so. But first we must ask, what is a \textbf{computer} ? Firstly, you must be able to keep a state. Secondly, we must be able to accept an input. A machine based on the Turing model can be used to do many thing, it's a generalist. 

\subsection{von Neumann Architecture}

The von Neumann Architecture is more or less what we have been using to even nowadays, it was revolutionary in its time, from input to datapath to memory to processing to output. 

\subsection{Common Function Units}

\begin{enumerate}
    \item Primary memory 
    \item Secondary memory 
    \item Cache memory
\end{enumerate}

\begin{enumerate}
    \item Input 
    \item Output
\end{enumerate}

All of them are connected by the buses and the controller. 

\section{Layered Views of Computers}
\begin{enumerate}
    \item Applications: Applications are a set of algorithms
    \item Algorithms: Algorithms are described with a language 
    \item Language: Language is an abstraction for the ISA for human
    \item Instruction Set Architecture: The ISA are a way for us to interact with microarchitecture  
    \item Microarchitecture 
    \item Circuits 
    \item Transistors: We are not talking about this
\end{enumerate}

Our main concerns are ISA to circuits. 

So, what is the difference between a calculating machine and a computer? \emph{A computer can be programmed. A calculator will always be a calculator while a computer can be anything. Therefore, the computer is a turing machine.}

